\environment env_whitepaper

\starttext

\title{Modeling of Arctic Ice: Whitepaper \hfill \tf\color[black]{David Li <li.davidm96@gmail.com>}}

\startcolumns[n=2,distance=0.2cm]
  \setupalign[flushleft,verytolerant]
  \startsection[title={Importance}]
    As climate change inexorably alters the climate of this planet,
    understanding the effects of such change has become more important than
    ever before. In particular, research has shown that Arctic ice can have
    a significant effect on weather in North America and Europe and may have
    played a role in recent severe winter storms \cite{Greene2012}.
  \stopsection

  \startsection[title={Background}]
    The Arctic Ocean is covered by floating ice termed Arctic ice; the
    thickness and coverage of this ice varies throughout the year and over
    time \cite{Polyak2010}. The ice affects weather by means of the Arctic
    Oscillation, an atmospheric system \cite{Greene2012}. The less ice
    present, the more heat the ocean absorbs during the summer in a process
    termed the \quotation{ice-albedo feedback mechanism}. This heat is
    released into the atmosphere via evaporation in the autumn, increasing
    atmospheric pressure and moisture and weakening the polar vortex and jet
    stream currents. The former contains cold air and the latter controls
    the persistence of that air in lower latitudes.

    The ice can be modeled with qualitative models such as the one described
    in Abbot et.\ al.\ 2011 \cite{Abbot2011}. Mathematical analysis of such
    models can reveal bifurcations, or qualitative changes in behavior; such
    findings can be applied to predict similar changes in real-world
    systems. One such bifurcation is termed a Hopf bifurcation and is of
    particular note because when a system contains slowly varying
    bifurcation parameters, the qualitative change in behavior will be
    delayed and occur later than expected mathematically \cite{Baer1989}.
  \stopsection

  \startsection[title={What’s Not Known}]
    Hopf bifurcations have not been applied to shorter-term Arctic ice
    research; it is not known whether one exists, nor have researchers
    searched for them in qualitative models as has been done for other
    bifurcation types. For instance, Abbot’s study only considered
    saddle-node bifurcations \cite{Abbot2011}. And in particular, greenhouse
    gas levels, which steadily rise, may serve as a slowly varying
    bifurcation parameter.
  \stopsection

  \startsection[title={Question}]
    As the presence of bifurcations in climate models would greatly impact
    the conclusions drawn from those models, and Hopf bifurcations have not
    yet been applied to these models, then naturally this question follows:
    how does the level of greenhouse gas as a slow parameter (one that
    continuously varies at a slow rate) affect the behavior of an Arctic sea
    ice model, and how can those conclusions be applied to a general climate
    model?
  \stopsection

  \startsection[title={Required Resources}]
    Due to the nature of the qualitative model, simulation is certainly
    doable on commodity computer hardware. Thus, no special equipment is
    necessary to investigate this question. However, being able to consult
    with a qualified climate scientist would help in interpreting the
    results of the model and making better conclusions.
  \stopsection

  \startsection[title={Impact}]
    A positive result (i.e. the presence of a Hopf bifurcation) would signal
    the possibility of a major change in the behavior of Arctic sea ice, a
    potentially worrying scenario in terms of the effect on weather,
    particularly winter weather. Meanwhile, a negative result would
    confirm that no such change is likely to occur if current trends
    continue. In either case, the results would help scientists modeling and
    predicting the behavior of Arctic sea ice, which in turn would affect
    the development of weather models incorporating that information;
    policymakers could also use the data to better make decisions on
    safeguards and preparation for climate change.
  \stopsection
\stopcolumns
\page[yes]
\startsection[title={References}]
  \setupalign[flushleft]
  \indenting[no]
  \placebibliography{summary.bib}
\stopsection

\stoptext