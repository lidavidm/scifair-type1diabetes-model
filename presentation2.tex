\environment env_presentation

\starttext

\title{Modeling of Arctic Ice: Redux}

{\startalignment[center] \ssa\sc

  David Li

  Dr.\ McKelvy

  18 Sep 2013

\stopalignment}

\startslide[title={Importance of the Field}]
% 2 dash, 5 circle
  \startitemize[2]
  \item Weather patterns in NA and Europe
  \item Interrelationships between ocean, atmosphere, Arctic systems
  \item Meteorologists \& general public
  \stopitemize
\stopslide

\page
\title{Review of Literature}

\startslide[title={\em Eigenvalues, Eigenvectors, and Differential Equations}]
  \cite{Cherry2009}
  \startitemize[2]
    \item Eigenvalues tell us about the solutions to a system of ODEs
    \item Real: eigenvectors point in asymptotic/repellent directions
    \item Imaginary: rotations
  \stopitemize

  \page[yes]

  \startformula
    \frac{dR}{dt} = aJ, \,\, \frac{dJ}{dt} = -bR, \,\, λ = \pm i \sqrt{ab}
  \stopformula
  \midaligned{\externalfigure[eigenvalue_example][height=300pt]}
\stopslide

\startslide[title={\em Introduction to bifurcation theory}]
  \cite{Crawford1989}
  \startitemize[2]
    \item Bifurcation: qualitative change due to a parameter change
    \item Dynamical system: set of differential equations, often used to
      model a physical system
    \item Changes: (dis)appearance of an equilibrium, periodic orbit
  \stopitemize
\stopslide

\startslide[title={\em Physics 161: Introduction to Chaos}]
  \cite{Cross2000}
  \startitemize[2]
  \item Stationary bifurcation: single real eigenvalue passes through $0$
  \item Hopf bifurcation: complex conjugate pair of eigenvalues passes
    through imaginary axis
  \stopitemize
  \midaligned{\externalfigure[hopf_example][height=300pt]}
\stopslide

\startslide[title={\em Precision and representation issues}]
  \cite{Mpmath2013}
  \startitemize[2]
  \item Numerical error causes:
    \startitemize[5]
    \item Rounding/cancellation (finite precision)
    \item Truncation (approximations to infinite sequences/continuous functions)
    \stopitemize
  \item {\tt mpmath}: library for floating-point arithmetic
  \item \quotation{In general, mpmath only guarantees that it will use at
    least the user-set precision to perform a given calculation}
  \item \quotation{The user may have to manually set the working precision
    higher than the desired accuracy for the result, possibly much higher.}
  \stopitemize

  Representation:

  \startformula
    mantissa \times 2^{exponent}
  \stopformula

  mantissa \& exponent are arbitrary-precision integers
\stopslide

\startslide[title={Current State of Knowledge}]
  \startitemize[2]
  \item Differential equations (dynamical systems) are used to model
    real-world phenomena
  \item Various mathematical tools exist to help analyze these models
  \item Bifurcations are interesting as they represent changes in behavior
    (\quotation{tipping points})
  \item Numerical error can be a problem when computing
  \stopitemize
\stopslide

\startslide[title={Outstanding Questions}]
  \startitemize[2]
  \item How do these mathematical concepts apply to real-world research?
    (particularly Hopf bifurcations)
  \stopitemize
\stopslide

\startslide[title={Research Question}]
{\ssa
  How does the level of greenhouse gas as a slow parameter (one that
  continuously varies at a slow rate) affect the behavior of an Arctic sea
  ice model, and how can those conclusions be applied to a general climate
  model?
}
\stopslide

\startslide[title={Proposed Procedure}]
  \startitemize[n]
  \item Implement model (Python)
    \startitemize[2]
    \item Requires RK4 solver
    \item Uses {\tt mpmath} for arbitrary-precision calculations
    \stopitemize
  \item Run model with varying parameter
    \startitemize[2]
    \item Model described in Abbot, 2011
    \item Some parameters updated
    \item Bifurcation parameter ($ε$) will be greenhouse gas level
    \stopitemize
    \page[yes]
  \item Generate diagrams
    \startitemize[2]
    \item Software: XPP AUTO
    \item (Research needed: can AUTO use data from an external program?)
    \stopitemize
  \item Analyze
    \startitemize
    \item The hard step
    \stopitemize
  \stopitemize

  \midaligned{\externalfigure[xpp][height=300pt]}
\stopslide

\startslide[title={Resources Needed}]
  \startitemize[2]
  \item No facilities needed
  \item No equipment needed
  \item Qualified scientist: Dr.\ Baer may be able to assist in verifying
    calculations
  \item Critical component: understanding of climate models
  \stopitemize
\stopslide

\BibFile{summary.bib}

\stoptext