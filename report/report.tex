% Page

\setuppapersize[letter]
\setuplayout[width=6.5in,height=10in,topspace=0.5in,backspace=1in,
  header=0.5in,footer=0.5in,bottomspace=0.5in]

% Typography
\definefontfeature
  [default]
  [default]
  [protrusion=quality,expansion=quality]

\setupalign[hz,hanging]
\definefontfamily[mainface][serif][TeX Gyre Pagella]
\definefontfamily[mainface][math][TeX Gyre Pagella Math]
\definefontfamily[mainface][mono][Consolas][scale=0.9,features=none]

\setupbodyfont[mainface,12pt]
\setupinterlinespace[1.3em]

\setupindenting[next]
\setupitemgroup[itemize][indentnext=no]
\indenting[yes, 0.5in]

\definestructureresetset[default][1,1,1][1]
\setuphead[sectionresetset=default]
\setuphead[title][style=\bfa,after={},alternative=middle]
\setuphead[section][before={\blank[0.5em]},after={\blank[0.5em]},aligntitle=yes]
\setuplabeltext[en][appendix=Appendix~]
\definehead[appendix][title]
\definehead[subappendix][appendix]
\def\Appendix#1{Appendix #1.}
\setuphead[appendix][alternative=right,conversion=A,before={},after={},page=yes,numbercommand=\Appendix,style={\tfa},number=yes,incrementnumber=yes]
\setuphead[subappendix][style={\tf\em},number=no,page=no]

\setupitemgroup[itemize][joinedup,nomargin]

\setupcombinedlist[content][list={section,appendix},alternative=c,]
\setupalign[tolerant]

\input mixin_bibliography
\input gnuplot
\input plots

\def\showplot#1{
  \useGNUPLOTgraphic[#1]
}

\def\placeplot#1{
  \placefigure[here]{}{
    \useGNUPLOTgraphic[#1]
  }
}

\def\infigure[#1]{\in{figure}{}[#1]}

\starttext

\indenting[no]
\title{Enhancing Theoretical Understanding of the Onset of Type 1 Diabetes}

\startalignment[middle]
  David Li

  Basha High School

  5990 South Val Vista Drive, Chandler, Arizona 85249

  \blank[line]

  Abstract
\stopalignment

  Type 1 diabetes, or autoimmune diabetes, is a disease where the body’s
  immune system destroys insulin-producing pancreatic beta cells.  Previous
  experiments in NOD (non-obese diabetic mice) revealed that the level of
  immune cells in the weeks before symptoms appeared followed a
  characteristic oscillatory or \quotation{spiking} pattern.  This
  experiment enhances the mathematical modeling of this phenomenon by
  analyzing one parameter, the level of beta cells, as it varies
  continuously (rather than leaving it constant), which, based on research
  in systems in other fields, can reveal unexpected behavior.  The
  experiment found the oscillations indicative of diabetes would either
  begin earlier or later than expected mathematically, implying that the
  disease may be detectable earlier, potentially allowing more time for
  treatment for those affected.

  \blank[line]

\placecontent

\indenting[yes]

\startsection[title=Introduction]


  In type 1 diabetes, also known as autoimmune diabetes, the body’s own
  immune system attacks and destroys insulin-producing pancreatic beta
  cells, leading to an insulin shortage and causing symptoms
  \cite{PubMed2013}. Currently, the causes and cures are largely unknown
  \cite{Daneman2006}. However, previous experiments have shown that in NOD
  (non-obese diabetic) mice, a standard model for diabetic research, the
  level of T cells (a specific type of immune cell) fluctuates cyclically in
  the weeks leading up to the appearance of symptoms, as depicted in
  \in{figure}{, top}[comb:introduction]: after the \quotation{spikes} occur
  (black), the percentage of diabetic mice increases dramatically (in gray)
  [\cite{Trudeau2003} cited in \cite{Mahaffy2007}]. To better understand
  the mechanism underlying these oscillations, Mahaffy and
  Edelstein-Keshet constructed a mathematical model of the
  immune–pancreas system. One parameter in the model is the level of
  pancreatic beta cells, which slowly decreases over time; at a certain
  level, the fluctuations described experimentally occur, and then
  diabetes appears \cite{Mahaffy2007}.


  % \placefigure[left][fig:mahaffy-t-cells]{From \cite{Mahaffy2007};
  %   experimental measurement of T cell levels in NOD mice over time, with
  %   diabetes occurring after
  %   \quotation{spikes}}{\externalfigure[experiment][width=3.5in]}
  % \placefigure[left][fig:hopf-example]{Example of a Hopf
  %   bifurcation.}{\externalfigure[bifurcation][width=2in]}
  \placefigure[left][comb:introduction]{
    Top: From \cite{Mahaffy2007}; experimental measurement of T cell
    levels in NOD mice over time, with diabetes occurring after
    \quotation{spikes}. Below: Example of a Hopf bifurcation.
  }{
    \startcombination[1*2]
      {\externalfigure[experiment][width=3.5in]}
      {\externalfigure[bifurcation][width=2in]}
    \stopcombination
  }

  \indentation Originally, the researchers analyzed the model’s behavior at
  various constant parameter values \cite{Mahaffy2007}, searching for {\em
    bifurcations}, or qualitative changes in behavior that occur when a
  parameter reaches a certain threshold \cite{VanVoorn2006}.  For instance,
  a system may be constant at one parameter value; if the parameter reaches
  a certain level, the system may then oscillate between two values (see
  \in{figure}{}[comb:introduction]) \cite{VanVoorn2006}. This experiment
  will improve the diabetes model by applying research demonstrating that in
  certain systems, slowly varying the parameter value while the model runs
  changes the qualitative nature of the system \cite{Baer1989}. This more
  accurately reflects what occurs biologically, as the original paper
  explicitly states the parameter continuously slowly falls. To summarize:
  for the original {\em static} analysis, the authors re-ran the model
  multiple times, each time setting the parameter to a fixed value. For the
  {\em continuous} analysis here, the parameter starts at a given value and
  then continuously decreases over time. Thus, this experiment will help
  scientists and medical professionals better understand and apply the
  theoretical results of Mahaffy and Edelstein-Keshet’s work as well as the
  experimental results they cited in predicting and understanding the onset
  of type 1 diabetes and the behavior of the immune system in this disease.

  Furthermore, when varying the parameter slowly, previous research noted
  that numerical precision greatly affected the results; if the calculations
  truncated too many digits (e.g.\ by using \quotation{single-precision}
  numbers), the results simply came out wrong.  This experiment will address
  that with {\tt mpmath}, a modern library for arbitrary-precision
  calculations that allows calculations at any precision \cite{Mpmath2013}.

\stopsection

\startsection[title={Question \& Hypothesis}]
  {\bf Question}: How does treating the peptide clearance rate $\delta_p$
  (essentially, the level of pancreatic beta cells) as a {\em continuously}
  and slowly varying parameter affect the qualitative behavior of the scaled
  reduced immune model developed by Mahaffy and Edelstein-Keshet, and how
  can these findings be applied to understanding and predicting the onset of
  type 1 diabetes?

  \noindentation {\bf Hypothesis}: If the model for the level of immune
  cells in the weeks before the onset of type 1 diabetes is analyzed with
  both a continuously varying and a static peptide clearance rate $δₚ$, then
  in the former analysis, the oscillations present in the original model
  that indicate the onset of diabetes will begin later than in the latter
  model because previous research has shown such behavior is delayed in
  other systems when similarly analyzed with a continuously varying
  parameter.
\stopsection

\startsection[title={Experimental Materials}]
  \startitemize[1,joinedup]

    \startitem Computer (x86-64 architecture running a Linux distribution)

      \indenting[no] The exact specifications and the operating system do not
      matter, as the software used runs on all platforms. A relatively recent
      (past 5 years or so) computer will suffice.
    \stopitem

    \startitem Computer Software
      \startitemize[2,joinedup]
      \item Python 3.3.3 with {\tt mpmath}
        (\hyphenatedurl{http://www.python.org},
        \hyphenatedurl{http://www.mpmath.org}), used to calculate the data for
        the plots.
      \item XPP AUTO 7.0 (http://www.math.pitt.edu/\textasciitilde
        bard/xpp/xpp.html), used to calculate the data for the bifurcation
        diagrams.
      \item \CONTEXT \ 2014.01.03
        (\hyphenatedurl{http://wiki.contextgarden.net}), used to generate
        the plots.
      \item Gnuplot 4.6.4 (http://www.gnuplot.info), used to generate
        plots.
      \stopitemize
    \stopitem

    \startitem Project Source Code

      \indenting[no]
      The source code for this project can be downloaded or checked out via
      {\tt git} from \hyphenatedurl{https://github.com/lidavidm/scifair-type1diabetes-model}.
    \stopitem
  \stopitemize

  Here are the version numbers of the software programs as reported
  themselves:
  \setuptyping[before=,after=]
  \starttyping
% python --version
Python 3.3.3
% ./xppaut -version
XPPAUT Version 7.0
% python -c "import mpmath; print(mpmath.__version__)"
0.17
% gnuplot --version
gnuplot 4.6 patchlevel 4
% context --version
mtx-context     | current version: 2014.01.03 00:40
  \stoptyping
\stopsection

\startsection[title={Experimental Procedures}]
  {\em These procedures represent a high-level overview; see \about[app:dp]
    on \at{page}[app:dp] for details.}

  \noindentation XPP AUTO, used in the original paper, is used to compute a
  bifurcation diagram:

  \showplot{bifurcation_xpp_simple}

  This shows, at a particular level of beta cells, possible states of the
  model, or the {\em maximum and minimum values of $A$ over time for a trial
    run of the model with the particular parameter level indicated on the
    x-axis}. Thus, it shows at what {\em static} parameter levels diabetes
  can occur; the higher the T cell level, the more beta cells can be
  destroyed. On a steady state, the system remains at a fixed level of T
  cells; in contrast, the level oscillates on a periodic state. If a state
  is steady, then introducing a small deviation will result in the system
  returning to the original state, while an unstable state will move towards
  another state. The points labeled \quotation{HB} are {\em Hopf
    bifurcations}, points at which the system changes qualitative
  behavior—from stable to periodic or vice-versa.

  Now, for the {\em continuous} analysis, the model starts at a particular
  $a_{15}$ (= 1/beta cell) level, which varies linearly over time. Because
  this variation is linear, at each time the $a_{15}$ level is unique, and
  thus the $A$ vs $t$ plot of this model can be overlaid on the bifurcation
  diagram as a $A$ vs $a_{15}$ plot, allowing them to be compared (see the
  diagrams below).

  The model equations are
  \startformula\startalign \frac{dA}{dt} \NC = \NC(a₆+a₇M)f₁(p) - a₈A - a₉A²
    \;\;\;\;\; \frac{dM}{dt} = a_{10}f₂(p)A-f₁(p)a₇a_{16}M - a_{11}M \NR
    \frac{dE}{dt} \NC = \NC a_{12}(1-f₂(p))A-a_{13}E \;\;\;\;\; p =
    \frac{a_{14}}{a_{15}}EB \;\;\;\;\; f₁ =
    \frac{p^{a_1}}{{a_2}^{a_1}+p^{a_1}} \;\;\;\;\; f₂ =
    \frac{{a_4}{a_5}^{a_3}}{{a_5}^{a_3}+p^{a_3}}\NR
  \stopalign \stopformula with $ a_{15} = δₚ = \mfunction{constant} $ for
  the bifurcation diagram or $ \dfrac{da_{15}}{dt} = \dfrac{dδₚ}{dt} =
  \mfunction{rate~of~change} $ for the continuous analysis\footnote{The rate
    of change depends on the parameter range in consideration (see the
    appendix).}. Here, $A$ represents the level of activated T cells, $B$
  the level of pancreatic beta cells, $E$ the effector T cells, $M$ the
  memory T cells, and $p$ the peptide level. (The fluctuation of activated T
  cells matters the most for the onset of diabetes.) In the model,
  pancreatic beta cells that undergo apoptosis (cell death) generate
  peptide; in lymph nodes, T cells become activated and
  \quotation{recognize} this particular peptide. These cells then become
  either effector cells, which destroy more pancreatic cells, thus producing
  more peptide, or memory cells, which encounter the peptide and trigger an
  immune response.

  Because $B$ and $δₚ$ are constant, an increase in the latter is equivalent
  to a decrease in the former (and vice-versa), so the latter was increased
  in Mahaffy’s analysis, which is equivalent to a decrease in beta cell
  levels \cite{Mahaffy2007}.  Additionally, a 4\high{th} order Runge-Kutta
  ODE solver written in Python is also used to compute the numerical
  solution in the second case. Runge-Kutta is a standard algorithm for
  solving ODEs; it is used here because Dr.\ Baer’s work noted that the
  continuous analysis could lead to numerical precision errors, and the
  Python implementation allows for arbitrary-precision computations
  \cite{Baer1989}\cite{Mpmath2013}\cite{Gonsalves2009}.

  This experiment will also generate model-over-time diagrams, which are
  overlaid over the bifurcation diagram to generate a \quotation{combined
    diagram}. Since the parameter will continuously and linearly vary over
  time, a correspondence between the points of the two graphs exists (i.e.\
  for the model-vs-time graph, at a certain time, the parameter will be at a
  unique particular value in the X-range of the bifurcation diagram), and
  therefore the two modes of simulation can be compared – the goal of this
  experiment. (See the appendix for the implementation details of processing
  the data and generating the diagrams.)
\stopsection

\startsection[title={Results \& Discussion}]
  \placefigure[force][fig:bc12]{Parameter range 0.1 to 2 over 200 days}%
              {\showplot{bifurcation_combined_12}}
  \placefigure[force][fig:bc12l]{Parameter range 0.1 to 2 over 1000 days}%
              {\showplot{bifurcation_combined_12_long}}
  \placefigure[force][fig:bc21]{Parameter range 2 to 0.1 over 200 days}%
              {\showplot{bifurcation_combined_21}}
  \placefigure[force][fig:bc0523]{Parameter range 0.5 to 2.3 over 200 days}%
              {\showplot{bifurcation_combined_0523}}
  \placefigure[force][fig:bch2f]{Demonstration of ramping of system through the second Hopf bifurcation (increasing parameter)}%
              {\showplot{bifurcation_combined_h2_forward}}
  \placefigure[force][fig:bch2]{Demonstration of ramping of system through the second Hopf bifurcation (decreasing parameter)}%
              {\showplot{bifurcation_combined_h2}}

  To read the plots, note that each contains two graphs: the continuous
  model, in orange, and the bifurcation diagram, in blue/green/red. The
  latter shows, at a particular \math{\delta_p}, the level of T cells the
  static model \quotation{settles into} over time. If multiple values are
  shown for a particular \math{\delta_p}, then the system oscillates between
  those values over time (and in this model, oscillations imply symptoms of
  diabetes \cite{Mahaffy2007}). Meanwhile, the continuous model (orange),
  which is the focus of this experiment, shows T cell level vs.\ time; since
  \math{\delta_p} varies linearly with time, each \math{\delta_p}
  corresponds to exactly one unique time value. The arrow in the upper right
  indicates the direction of time.

  In the original experiments with mice, the appearance of symptoms of
  diabetes corresponded with cyclic fluctuations in the level of T cells
  \cite{Trudeau2003}. Looking at \infigure[fig:bc12], as the level of T
  cells decreases over time in the continuous model, some oscillations
  appear before the point they are expected to (the {\em Hopf bifurcation}
  point, labeled \quotation{HB}). Given the large error bars in the original
  experiments, however, such small oscillations may not be of note and may
  explain why only a few oscillations were found in the mice. Indeed, if the
  system is started closer to the Hopf bifurcation in \infigure[fig:bc0523],
  the oscillations begin at the expected time. When the same amount of beta
  cell destruction as in \infigure[fig:bc12] is spread out over a longer
  timeframe—1000 days rather than 200 (the experiments covered roughly 200
  days)—the oscillations occur far later than they should, as seen in
  \infigure[fig:bc12l]. Thus, theoretically, a way to prevent beta cell
  destruction would only delay the onset of symptoms. Meanwhile, in
  \infigure[fig:bc21], the beta cell level increases at a rate that outpaces
  the destruction caused by the immune system. In this scenario, the immune
  system still displays the response characteristic of the onset of
  diabetes—but as beta cell level is increasing, this response does not
  matter.

  No cyclic behavior occurred at all for the second Hopf bifurcation, as
  demonstrated by \infigure[fig:bch2f] and \infigure[fig:bch2]. These
  diagrams can be interpreted best in terms of \math{\delta_p}, the peptide
  clearance rate. In the model, pancreatic cells die and produce peptide,
  some of which is not cleared away and produces peptide, triggering an
  immune response—\math{\delta_p} controls how much peptide is left over, in
  essence. For \infigure[fig:bch2f], where the beta cell level decreased,
  killing cells causes no oscillatory response because the high clearance
  rate prevents the immune system from \quotation{noticing} the peptide, in
  effect. Similarly, in \infigure[fig:bch2], where beta cells regenerate,
  the presence of immune cells implies that some beta cells die, but again,
  the high clearance rate prevents any adverse reaction.

  \setupcaptions[location=right]
  \placefigure[here]{Plot of eigenvalues of system versus parameter
    value. The areas under the curve before and after the bifurcation do not
    equal.}{\externalfigure[eigenvalue_plot][scale=600]}

  To determine when exactly the model \quotation{destabilizes} and begins
  oscillations, the eigenvalues of the system of ODEs can be plotted versus
  the parameter (i.e.\ versus time); generally such systems destabilize when
  the positive area under this plot after the bifurcation equals the area
  under the plot before \cite{Baer1989}. (By definition, a Hopf bifurcation
  is when the real part of a pair of complex conjugate eigenvalues crosses
  through zero as the parameter changes \cite{VanVoorn2006}.) However, as
  evidenced by the above plot, these areas do not equal, and thus, the
  condition does not apply to this model.

  As previously described, this experiment performed calculations both using
  XPP and with Python and {\tt mpmath} to avoid error with numerical
  roundoff. This results only shows results from XPP as in all cases studied
  herein, the calculations matched qualitatively (results were not compared
  at a numerical level). However, {\tt mpmath} can help in cases
  where the necessary numerical precision exceeds that the system can
  provide. For instance, some non-standard implementations of Fortran
  provide a quadruple-precision type {\tt real*16} \cite{GFortran2014} with
  113 bits of significand precision \cite{IEEE7542008}. {\tt mpmath}, in
  contrast, can provide any precision necessary (though being implemented in
  software, is slower). Consider a hypothetical immune system where
  \math{d\delta_p/dt} is \math{0.0002}; running at double precision, the
  system destabilizes prematurely. As shown by \in{figure}[fig:precision] on
  \at{page}[fig:precision], using \quotation{double-precision} numbers leads
  to the system incorrectly destabilizing prematurely. With {\tt mpmath} set
  to a higher precision, the system behaves as expected.

  \setupcaptions[location=bottom]
  \placefigure[here][fig:precision]{Double-precision (in Python, left; in
    XPP, center) leads to incorrect results. The precision of Python’s float
    may vary by platform; as reported by {\tt sys.float_info} the bits of
    precision here is 53.}{
    \startcombination[3*1]
      {\externalfigure[precision_double][scale=400]}{}
      {\externalfigure[precision_xpp][scale=400]}{}
      {\externalfigure[precision_100][scale=400]}{}
    \stopcombination
  }

\stopsection

\startsection[title={Conclusion \& Future Research}]
  The results both confirm and refute the hypothesis—depending on the
  timescale, the oscillations that are indicative of the onset of diabetes
  may start either earlier or later than expected. Taking into account the
  error bars on the original data, the simulations here still show the
  characteristic pattern of low T cell levels that then transition into
  oscillatory \quotation{spiking}. Thus, this confirms the veracity of the
  original model under a more biologically accurate continuous
  analysis. However, the results herein further suggest that 1) with
  sensitive enough tests, the oscillations characteristic of the disease may
  be detectable earlier than expected, and that 2) slowing the rate of beta
  cell destruction may delay, but not prevent, those
  oscillations. Therefore, researchers interested in completely preventing
  this disease may want to investigate other variables of the model, and
  being able to theoretically detect the onset of type 1 diabetes earlier
  than expected before would allow more time for treatment and hopefully a
  higher quality of life for those affected.

  Experimental error in this model may result from round-off due to finite
  numerical precision. To address this, {\tt mpmath} with Python allowed the
  calculations to be conducted at arbitrary precision; this data was then
  compared with the result from XPP. Multiple trials were not necessary as
  the calculations were deterministic. (Internally, {\tt mpmath} represents
  values with an arbitrary-length integer mantissa and exponent; the library
  also guarantees the correctness of basic arithmetic operations, the only
  ones used here \cite{Mpmath2013}.)


  In this analysis, the level of beta cells varies linearly, but in reality,
  beta cells regenerate. Thus, extending the system with a model of beta
  cell regeneration would make it more biologically accurate. For instance,
  models of type 2 diabetes often account for this and could serve as a
  starting point for research [Dr.\ Jiaxu Li, University of Louisville,
    personal communication, December 6, 2013]. Additionally, conducting
  further experiments in mice to obtain more data on the oscillations would
  help fine-tuning the model; note both the large error bars in the original
  data (see introduction) and the parameter sensitivity of this model
  \cite{Mahaffy2007}.
\stopsection

\startsection[title={References}]
  {\setupalign[nothyphenated]
  \placebibliography{summary.bib}
  }
\stopsection
\setupalign[hyphenated]
\startsection[title={Acknowledgments}]

  Thanks goes to Dr.\ Michael McKelvy, Dr.\ Stephen Baer, Dr.\ Joseph
  Mahaffy, and Dr.\ Jiaxu Li for all their help, and to my parents for
  driving me everywhere.

\stopsection
\startappendix[title={Detailed Procedures}, reference=app:dp]

  {\em Note: familiarity with the Linux command line is assumed.}

\startitemize[n,joinedup]
\item Download the necessary experiment software and experiment files
  (location TBD).
\item Extract the experiment files to a directory.
\item Open a shell and {\tt cd} to the directory containing the experiment
  files.
\startitem To generate plots of the system over time:
  \startitemize[a]
  \item Run {\tt python3 rk4_ode.py} to generate the data via the Runge-Kutta
    method.
  \item Run {\tt ./time_diagrams_xppaut.sh} to generate the data via XPP
    AUTO.
  \item Run {\tt context time_diagrams.tex} to generate a PDF containing the
    plots.
  \stopitemize
\stopitem
\startitem Then, to generate the plots of the system vs the bifurcation
  parameters (the peptide clearance rate, $δₚ$; fraction of memory T cells
  produced, $a$; T cell competition parameter $ε$, and peptide level for ½
  max memory cells, $k₂$):
  \startitemize[a]
    \startitem[item:auto-steps] Run {\tt ../path_to_xpp/xppaut -runnow mk_static.ode} to
      launch XPP AUTO. Bring up AUTO and save the points for the complete
      bifurcation diagram in {\tt mk_static.dat}. If other parameters are to
      be analyzed, also save the diagrams for those cases.

      \startitemize[i]
      \item Press {\tt f a} to bring up AUTO.
      \item Press {\tt f l} and choose {\tt mk_static.ode.auto} to load the
        parameters for the bifurcation diagram.
      \item Press {\tt r s} to generate the diagram.
      \item Press {\tt g}, {\tt <TAB>} over to the bifurcation point
        (labeled {\tt HB} on the bottom), press {\tt <ENTER>} to select it,
        and then {\tt rp} to generate the branch.
      \item Repeat for each bifurcation point.
      \item Press {\tt f a} and save the points in {\tt mk_static.dat}.
      \item Close XPP AUTO ({\tt Control-C} the program in the terminal).
      \stopitemize
    \stopitem
  \item Repeat \in{step}[item:auto-steps] for {\tt mk_static_epsilon.ode}
    with {\tt mk_epsilon.ode.auto} and {\tt mk_static_k2.ode} with {\tt
      mk_k2.ode.auto}, and then for {\tt mk_continuous_epsilon.ode} and {\tt
      mk_continuous_k2.ode} with the corresponding {\tt .auto} files. Name
    the result file after the parameter and {\tt .ode} file (e.g. {\tt
      mk_continuous_epsilon.dat}, {\tt mk_static_k2.dat}).
  \item Run {\tt python3 bifurcation_diagrams.py} to process the plot data.
  \item Run {\tt context bifurcation_diagrams.tex} to generate a PDF
    containing the plots.
  \stopitemize
\stopitem
\stopitemize

\startsubappendix[title={Parameters for XPP AUTO and Python}]
  The parameters in the following table are {\em not} parameters for the
  model, but rather for the programs used to generate the model data. All
  parameters not listed were left at default values.

  \startTABLE[split=repeat]
    \setupTABLE[r][first][style=bold]
    \setupTABLE[c][1][width=.7\textwidth]
    \setupTABLE[c][2][width=.3\textwidth]
    \bTABLEhead
    \NC Parameter \NC Value \NC\NR
    \eTABLEhead

    \bTABLEbody
    \NC Integration time (the time the system will be run for) \NC 200 days \NC\NR
    \NC Integration step (RK4, both Python and XPP AUTO) \NC 0.05 days \NC\NR
    \NC {\tt mpmath} Decimal Precision ({\tt mpmath.mp.dps}) \NC 600 \NC\NR
    \NC AUTO Par 1 \NC {\tt a15} \NC\NR
    \NC AUTO Hi-Lo Y-axis \NC {\tt A} \NC\NR
    \NC AUTO Hi-Lo Main Parameter \NC {\tt a15} \NC\NR
    \NC AUTO Hi-Lo ranges \NC X: $\left[0, 18 \right]$ Y: $\left[0, 3\right]$ \NC\NR
    \NC AUTO Par Max \NC 18 \NC\NR
    \eTABLEbody

  \stopTABLE
\stopsubappendix

\startsubappendix[title={Initial Conditions}]
For the model when processed over time:
\startformula
A = 0.5 \;\;\;\;\;
M = 0 \;\;\;\;\;
E = 1 \;\;\;\;\;
B = 1 \;\;\;\;\;
a_{15} = \mfunction{See~below}
\stopformula

For the model when generating bifurcation diagrams for the parameter
$a_{15}$:

\startformula
A = 0.01961524 \;\;\;\;\;
M = 2.802597\times10^{-45} \;\;\;\;\;
E = 0.006538414 \;\;\;\;\;
B = 1 \;\;\;\;\;
a_{15} = 0
\stopformula
\stopsubappendix

\startsubappendix[title={Parameter Ranges}]
  As the total time is 200 days, the parameter will be varied at a rate of $
  (δ_{p_2} - δ_{p_1})/200$ per day. These ranges were chosen based on a
  reading of Mahaffy’s paper \cite{Mahaffy2007}. Although the authors did
  not call out explicit parameter ranges, the ranges here represent either
  the domains of graphs given or are near bifurcation points they
  noted. Also, they tended to include a parameter value of $0$, which has
  been adjusted to $0.1$ here to avoid a division-by-zero error. XPP AUTO does
  not return an error when this occurs, but Python does, so the adjustment
  avoids conflicts and undocumented behaviors. This will not adversely
  affect the impact of this experiment; because the original analysis used
  static parameter values, the overall range of values does not matter as
  long as it includes the range used in the continuous analysis to be done
  here for comparison.

  \bTABLE
    \setupTABLE[r][first][style=bold]
    \setupTABLE[c][1][width=\textwidth]
    \bTABLEhead
    \bTR \bTD $\delta_p$, the peptide clearance rate \eTD \eTR
    \eTABLEhead

    \bTABLEbody
    \bTR \bTD 0.1 to 2   \eTD \eTR
    \bTR \bTD 2 to 0.1   \eTD \eTR
    \bTR \bTD 0.5 to 2.3 \eTD \eTR
    \bTR \bTD 16.31 to 0.31 \eTD \eTR
    \bTR \bTD 3.369 to 19.369 \eTD \eTR
    \eTABLEbody

  \eTABLE
\stopsubappendix

\stopappendix

\stoptext