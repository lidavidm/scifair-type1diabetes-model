% Page

\setuppapersize[letter]
\setuplayout[width=6.5in,height=10in,topspace=0.5in,backspace=1in,
  header=0.5in,footer=0.5in,bottomspace=0.5in]

% Typography

\definefontfamily[mainface][serif][TeX Gyre Pagella]
\definefontfamily[mainface][math][TeX Gyre Pagella Math]
\definefontfamily[mainface][mono][Consolas][scale=0.9,features=none]

\setupbodyfont[mainface,12pt]
\setupinterlinespace[1.3em]

\setupindenting[next]
\setupitemgroup[itemize][indentnext=no]
\indenting[yes, 0.5in]

\setuphead[title][style=\bfa,after={},alternative=middle]

\setupitemgroup[itemize][joinedup,nomargin]

\setupcombinedlist[content][alternative=c,]

\def\cite#1{[#1]}

\starttext

\indenting[no]
\title{Enhancing Theoretical Understanding of the Onset of Type 1 Diabetes}
David Li

\startalignment[middle]

  Abstract

  \blank[line]

  This is my abstract.

\stopalignment

\placecontent

\indenting[yes]

\startsection[title=Introduction]
  In type 1 diabetes, also known as autoimmune diabetes, the body’s own immune
  system attacks and destroys insulin-producing pancreatic beta cells, leading
  to an insulin shortage and causing symptoms \cite{PubMed2013}. Currently,
  the causes and cures are largely unknown \cite{Daneman2006}. However,
  previous experiments have shown that in NOD (non-obese diabetic) mice, a
  standard model for diabetic research, the level of T cells (a specific type
  of immune cell) fluctuates cyclically in the weeks leading up to the
  appearance of symptoms [\cite{Trudeau2003} cited in \cite{Mahaffy2007}]. To
  better understand the mechanism underlying these oscillations, Mahaffy and
  Edelstein-Keshet constructed a mathematical model of the immune–pancreas
  system. One parameter in the model is the level of pancreatic beta cells,
  which slowly decreases over time as the disease progresses; at a certain
  level, the fluctuations described experimentally appear \cite{Mahaffy2007}.


  \setupcaptions[location=right,width=7cm]
  \placefigure[here,nonumber]{Left: from \cite{Mahaffy2007}; experimental
    measurement of T cell levels in NOD mice over time, with diabetes
    occurring after \quotation{spikes}. Right: example of a Hopf bifurcation.}{\startcombination[2*1]
    {\externalfigure[experiment][height=8cm]}{}
    {\externalfigure[bifurcation][height=8cm]}{}
  \stopcombination}


  Originally, the researchers analyzed the model’s behavior at various
  constant parameter values \cite{Mahaffy2007}; in particular, they searched
  for {\em bifurcations}, or qualitative changes in behavior that occur when a
  parameter reaches a certain threshold \cite{VanVoorn2006}. For instance, a
  system may remain constant at one parameter value; if the parameter reaches
  a certain level, the system may then oscillate between two defined values
  (see diagram) \cite{VanVoorn2006}. This experiment will improve the model by
  applying research demonstrating that in certain systems, slowly varying the
  parameter value while the model runs can change the qualitative nature of
  the system \cite{Baer1989}. This more accurately reflects what occurs
  biologically, as the original paper explicitly stated that the parameter
  should continuously slowly fall. To summarize: for the original {\em static}
  analysis, the authors re-ran the model multiple times, each time setting the
  parameter to a fixed value. For the {\em continuous} analysis here, the
  parameter starts at a given value and then continuously decreases over
  time. Thus, this experiment will help scientists and medical professionals
  better understand and apply the theoretical results of Mahaffy and
  Edelstein-Keshet’s work as well as the experimental results they cited in
  predicting and understanding the onset of type 1 diabetes and the behavior
  of the immune system in this disease.
\stopsection

\startsection[title={Question \& Hypothesis}]
  How does treating the level of beta cells as a {\em continuously} varying
  slow parameter affect the qualitative behavior of the scaled reduced
  immune model developed by Mahaffy and Edelstein-Keshet, and how can those
  findings be applied to understanding and predicting the onset of type 1
  diabetes?

  If the model for the level of immune cells in the weeks before the onset
  of type 1 diabetes is analyzed with both a continuously varying and a
  static peptide clearance rate $δₚ$, then in the former analysis, the
  oscillations present in the original model will begin at a later time
  because research has shown this behavior is delayed in other models when
  analyzed with a continuously varying parameter.

\stopsection

\startsection[title={Experimental Materials}]
  \startitemize[1,joinedup]

    \startitem Computer (x86-64 architecture running a Linux distribution)

      \indenting[no] The exact specifications and the operating system do not
      matter, as the software used runs on all platforms. A relatively recent
      (past 5 years or so) computer will suffice.
    \stopitem

    \startitem Computer Software
      \startitemize[1,joinedup]
      \item Python 3.3.3 with {\tt mpmath}
        (\hyphenatedurl{http://www.python.org},
        \hyphenatedurl{http://www.mpmath.org}), used to calculate the data for
        the plots.
      \item XPP AUTO 7.0 (http://www.math.pitt.edu/\textasciitilde
        bard/xpp/xpp.html), used to calculate the data for the bifurcation
        diagrams.
      \item \CONTEXT \ 2014.01.03 (http://wiki.contextgarden.net), used to
        generate the plots and bifurcation diagrams.
      \item Gnuplot 4.6.4 (http://www.gnuplot.info), used to generate
        plots.
      \stopitemize
    \stopitem

    \startitem Project Source Code

      \indenting[no]
      The source code for this project can be downloaded or checked out via
      {\tt git} from this URL:

      \startalignment[middle]
        http://www.bitbucket.org/lidavidm/scifair-t1d-model
      \stopalignment
    \stopitem
  \stopitemize

  For reference, here are the version numbers of the software programs as
  reported themselves:
  \starttyping
% python --version
Python 3.3.3
% ./xppaut -version
XPPAUT Version 7.0
% python -c "import mpmath; print(mpmath.__version__)"
0.17
% gnuplot --version
gnuplot 4.6 patchlevel 4
% context --version
mtx-context     | ConTeXt Process Management 0.60…
mtx-context     | current version: 2014.01.03 00:40
  \stoptyping
\stopsection

\startsection[title={Experimental Procedures}]

  XPP AUTO, used in the original paper, is used to compute a bifurcation
  diagram:

  %\showplot{bifurcation_xpp_simple}

  This shows, at a particular level of beta cells, possible states of the
  model (i.e.\ level of T cells). The higher the T cell level, the more beta
  cells can be destroyed. On a steady state, the system remains at a fixed
  level of T cells; in contrast, the level oscillates on a periodic state. If
  a state is steady, then introducing a small deviation will result in the
  system returning to the original state, while an unstable state will move
  towards another state. The points labeled \quotation{HB} are {\em Hopf
    bifurcations}, points at which the system changes qualitative
  behavior—from stable to periodic or vice-versa.

  Now, for the {\em continuous} analysis, the model starts at a particular
  $a_{15}$ (= 1/beta cell) level, which varies linearly over time. Because
  this variation is linear, at each time the $a_{15}$ level is unique, and
  thus the $A$ vs $t$ plot of this model can be overlaid on the bifurcation
  diagram as a $A$ vs $a_{15}$ plot, allowing them to be compared (see the
  diagrams below).

  The model equations are
  \startformula\startalign \frac{dA}{dt} \NC = \NC(a₆+a₇M)f₁(p) - a₈A - a₉A²
    \;\;\;\;\; \frac{dM}{dt} = a_{10}f₂(p)A-f₁(p)a₇a_{16}M - a_{11}M \NR
    \frac{dE}{dt} \NC = \NC a_{12}(1-f₂(p))A-a_{13}E \;\;\;\;\; p =
    \frac{a_{14}}{a_{15}}EB \;\;\;\;\; f₁ =
    \frac{p^{a_1}}{{a_2}^{a_1}+p^{a_1}} \;\;\;\;\; f₂ =
    \frac{{a_4}{a_5}^{a_3}}{{a_5}^{a_3}+p^{a_3}}\NR
  \stopalign \stopformula
  with $ a_{15} = δₚ = \mfunction{constant} $ for the bifurcation diagram or $
  \dfrac{da_{15}}{dt} = \dfrac{dδₚ}{dt} = \mfunction{rate~of~change} $ for the
  continuous analysis. Here, $A$ is the level of activated T cells, $B$
  pancreatic beta cells, $E$ effector T cells, $M$ memory T cells, and $p$
  peptide level. In the model, beta cells that undergo apoptosis (cell death)
  generate peptide; then, T cells become activated, \quotation{recognizing}
  this peptide. These then become either effector cells, which destroy
  pancreatic cells and produce more peptide, or memory cells that encounter
  the peptide and trigger an immune response.

  A 4th order Runge-Kutta solver is used to solve the continuous
  system. Runge-Kutta is a standard algorithm for solving differential
  equations \cite{Gonsalves2009}, used here because the continuous analysis
  could lead to numerical precision errors \cite{Baer1989}, and this
  implementation allows for arbitrary-precision computations
  \cite{Mpmath2013}.
\stopsection

\startsection[title={Results \& Discussion}]

  To read the plots, note that each contains two graphs: the continuous
  model, in orange, and the bifurcation diagram, in blue/green/red. The
  latter shows, at a particular \math{\delta_p}, the level of T cells the
  static model \quotation{settles into} over time. If multiple values are
  shown for a particular \math{\delta_p}, then the system oscillates between
  those values over time (and in this model, oscillations imply symptoms of
  diabetes \cite{Mahaffy2007}). Meanwhile, the continuous model (orange)
  shows T cell level vs.\ time; since \math{\delta_p} varies linearly with
  time, each \math{\delta_p} corresponds to exactly one unique time
  value. The arrow in the upper right indicates the direction of time.

  In the original experiments with mice, the appearance of symptoms of
  diabetes corresponded with cyclic fluctuations in the level of T cells
  \cite{Trudeau2003}. Looking at the first plot, as the level of T cells
  decreases over time in the continuous model, some oscillations appear
  before the point they are expected to (the {\em Hopf bifurcation} point,
  labeled \quotation{HB}). Given the large error bars in the original
  experiments, however, such small oscillations may not be of note and may
  explain why only a few oscillations were found in the mice. When the same
  amount of beta cell destruction is spread out over a longer timeframe—1000
  days rather than 200 (the experiments covered roughly 200 days)—the
  oscillations occur far later than they should. Thus, theoretically, a way
  to prevent beta cell destruction would only delay the onset of symptoms.

\stopsection

\startsection[title={Conclusion \& Future Research}]
  The results both confirm and refute the hypothesis—depending on the
  timescale, the oscillations that are indicative of the onset of diabetes
  may start either earlier or later than expected. Taking into account the
  error bars on the original data, the simulations here still show the
  characteristic pattern of low T cell levels that then transition into
  oscillatory \quotation{spiking}. Thus, this confirms the veracity of the
  original model under a more biologically accurate continuous
  analysis. Furthermore, the results suggest that 1) with sensitive enough
  tests, the oscillations characteristic of the disease may be detectable
  earlier than expected, and that 2) slowing the rate of beta cell
  destruction may delay, but not prevent, those oscillations. Therefore,
  researchers interested in completely preventing this disease may want to
  investigate other variables of the model.

  In this analysis, the level of beta cells varies linearly, but in reality,
  beta cells regenerate. Thus, extending the system with a model of beta
  cell regeneration would make it more biologically accurate. For instance,
  models of type 2 diabetes often account for this and could serve as a
  starting point for research [Dr.\ Jiaxu Li, University of Louisville,
    personal communication, December 6, 2013]. Additionally, conducting
  further experiments in mice to obtain more data on the oscillations would
  help fine-tuning the model; note both the large error bars in the original
  data (see introduction) and the parameter sensitivity of this model
  \cite{Mahaffy2007}.
\stopsection

\startsection[title={References}]
\stopsection

\startsection[title={Acknowledgments}]
\stopsection

\stoptext