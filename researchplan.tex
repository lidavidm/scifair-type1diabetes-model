\environment env_essay
\indenting[no]
\setupitemgroup[itemize][indenting=no]
\setupwhitespace[small]
\setuphead[title][style=\tfb,after={},before={}]
\setuphead[section][number=no,style=\tfa,before={\blank[0.2cm]},after={\blank[0.1cm]}]
\setuphead[subsection][number=no,style=\tf\em,before={},after={}]

\input bibliography

\setupheadertexts[][David Li][David Li][]

\starttext

\title{Research Plan}

David Li

\startsection[title={Question}]
  How does treating the level of beta cells as a {\em continuously} varying
  slow parameter affect the qualitative behavior of the scaled reduced
  immune model developed by Mahaffy and Edelstein-Keshet \cite{Mahaffy2007},
  and how can those findings be applied to understanding and predicting
  type 1 diabetes?
\stopsection

\startsection[title={Goals and Hypothesis}]
  \startsubsection[title={Background Information}]
    Autoimmune diabetes, or type 1 diabetes, is a disease in which the
    immune system attacks insulin-producing pancreatic beta cells, leading
    to high blood glucose levels and other symptoms
    \cite{PubMed2013}. Furthermore, no screening test exists for this
    disease, which can only be diagnosed after symptoms appear
    \cite{PubMed2013}. Research with NOD (non-obese diabetic) mice has
    demonstrated that the onset of diabetes follows an elevated level of
    autoreactive T cells in the blood (which destroy beta cells);
    furthermore, the level of these cells followed a cyclic pattern over
    time \cite{Trudeau2003}. To better understand the mechanism underlying
    these oscillations, Mahaffy and Keshet constructed a mathematical model
    that explained the phenomenon as caused by a gradual decrease in beta
    cell level \cite{Mahaffy2007}.

    Mathematical analysis of such models can reveal bifurcations, or
    qualitative changes in behavior; such findings can be applied to predict
    similar changes in real-world systems. One type, termed a Hopf
    bifurcation, is of particular note because when the model parameter is
    continuously slowly varying, the oscillations this bifurcation normally
    causes will be delayed and will occur later than expected
    \cite{Baer1989}.  Mahaffy’s model has a Hopf bifurcation, which is
    responsible for the oscillating T cell levels, but the researchers {\em
      did not analyze it with a continuously varying parameter}, though they
    explicitly stated the parameter should behave this way.
  \stopsubsection
  \startsubsection[title={Goals}]
    This experiment aims to complete the analysis by looking into the
    behavior of the model when the level of pancreatic beta cells is treated
    as a continuously varying parameter. In particular, the analysis will
    focus on the behavior of the model with this modification compared to
    the behavior described in the original paper and the behavior of actual
    the biological system in NOD mice.
  \stopsubsection
  \startsubsection[title={Hypothesis}]
    If the model for the level of immune cells in the weeks before the onset
    of type 1 diabetes is analyzed with both a continuously varying and a
    static peptide clearance rate $δₚ$, then in the former analysis, the
    oscillations present in the original model will begin at a later time
    because research has shown this behavior is delayed in other models
    when analyzed with a continuously varying parameter.
  \stopsubsection
\stopsection

\startsection[title={Procedures}]
  See the attached procedures.

  \startsubsection[title={Risk and Safety}]
    As the experiment solely involves numerical computations, no safety
    risks are anticipated.
  \stopsubsection
  \startsubsection[title={Data Analysis}]
    After simulation, the experiment will result in a set of bifurcation
    diagrams as well as model-over-time diagrams.  The bifurcation diagram
    shows the maximum and minimum values of $A$ over time for a trial run of
    the model with the particular parameter level indicated on the
    x-axis. Thus, the bifurcation diagram shows at what {\em static}
    parameter levels diabetes can occur. For this experiment, a third
    diagram will be generated overlaying the model-over-time diagram for the
    {\em continuous} system onto the bifurcation diagram. Since the
    parameter will continuously and linearly vary over time, a
    correspondence between the points of the two graphs exists (i.e.\ for
    the model-vs-time graph, at a certain time, the parameter will be at a
    unique particular value in the X-range of the bifurcation diagram), and
    therefore the two modes of simulation can be compared – the goal of this
    experiment.
  \stopsubsection
\stopsection

\startsection[title={Bibliography}]
\inviscite{Mahaffy2007}
\inviscite{Baer1989}
\inviscite{Mpmath2013}
\inviscite{PubMed2013}
\inviscite{Gonsalves2009}

\placebibliography{summary.bib}
\stopsection

\stoptext
