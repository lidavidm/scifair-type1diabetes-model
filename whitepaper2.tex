\environment env_whitepaper

\starttext

\title{Modeling of T Cells in Autoimmune Diabetes: Whitepaper \hfill \tf\color[black]{David Li <li.davidm96@gmail.com>}}

\startcolumns[n=2,distance=0.2cm]
  \setupalign[hyphenated]
  \startsection[title={Importance}]
    Autoimmune diabetes, or type 1 diabetes, is a chronic disease that
    causes high blood glucose levels; this type of diabetes is an autoimmune
    disorder, where the immune system attacks healthy body tissue, leading
    to symptoms \cite{PubMed2013}. Furthermore, no screening test exists for
    this disease, which can only be diagnosed after symptoms appear
    \cite{PubMed2013}.

  \stopsection

  \startsection[title={Background}]
    In type 1 diabetes, pancreatic beta cells do not produce enough insulin,
    a hormone used by cells to take glucose from the blood; thus, sugar
    builds up and cells cannot use glucose for energy, leading to symptoms
    such as blurry vision and possibly complications such as blindness
    \cite{PubMed2013}. The lack of insulin stems from the destruction of the
    beta cells by the immune system. Research with NOD (non-obese diabetic)
    mice has demonstrated that the onset of diabetes follows an elevated
    level of autoreactive T cells in the blood (which destroy beta cells);
    furthermore, the level of these cells followed a cyclic pattern over
    time \cite{Trudeau2003}.  To better understand this mechanism underlying
    these oscillations, Mahaffy and Keshet constructed a mathematical model
    that explained the cause of the phenomenon as a gradual decrease in beta
    cell level \cite{Mahaffy2007}.

    Mathematical analysis of such models can reveal bifurcations, or
    qualitative changes in behavior; such findings can be applied to predict
    similar changes in real-world systems. One type, termed a Hopf
    bifurcation, is of particular note because when the model parameter is
    continuously slowly varying, the oscillations this bifurcation normally
    causes will be delayed and will occur later than expected
    \cite{Baer1989}.  Mahaffy’s model has a Hopf bifurcation, which is
    responsible for the oscillating T cell levels, but the researchers did
    not analyze it with a continuously varying parameter.

  \stopsection

  \startsection[title={What’s Not Known}]
    The creators of the model explicitly state the beta cell level slowly
    varies \cite{Mahaffy2007}, but did not analyze it by continuously
    varying the parameter; instead, they looked at the behavior with
    multiple static values of the level of beta cells. Since having a
    continuous parameter can change the qualitative behavior of a system,
    and thus affect the conclusions regarding the validity of a model,
    conducting this analysis matters – especially as it affects the Hopf
    bifurcation, which explains the oscillations that experimentally
    preceded the onset of diabetes \cite{Baer1989}\cite{Mahaffy2007}.
  \stopsection

  \startsection[title={Question}]
    As the accuracy of the model would impact the conclusions drawn and
    possible applications, then naturally this question follows: how does
    treating the level of beta cells as a {\em continuously} varying slow
    parameter affect the qualitative behavior of the immune model, and how
    can those findings be applied to understanding and predicting autoimmune
    (type 1) diabetes?
  \stopsection

  \startsection[title={Required Resources}]
    Due to the nature of the model, simulation is certainly doable on
    commodity computer hardware. Thus, no special equipment is necessary to
    investigate this question. However, being able to consult with a
    qualified scientist would help in verifying calculations, interpreting
    the results, and making better conclusions.
  \stopsection

  \startsection[title={Impact}]
    Analyzing the model with a continuously varying slow parameter could
    imply new physical behavior to search for; in particular, the ability to
    delay the onset of oscillations may predict a way to prevent the
    development of diabetes. Meanwhile, a more accurate model could improve
    the usefulness of this research in predicting diabetes. On the other
    hand, if the new model no longer matches the experimental evidence, that
    would indicate that it may not correctly model the actual behavior of
    the immune system, thus indicating to researchers that they should focus
    their efforts elsewhere.
  \stopsection
\stopcolumns
\page[yes]
\startsection[title={References}]
  \setupalign[flushleft]
  \indenting[no]
  \placebibliography{summary.bib}
\stopsection

\stoptext