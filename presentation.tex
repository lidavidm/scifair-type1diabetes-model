\environment env_presentation

\starttext

\title{Modeling of Arctic Ice}

{\startalignment[center] \ssa\sc

  David Li

  Dr.\ McKelvy

  21 Aug 2013

\stopalignment}

\startslide[title={Importance of the Field}]
% 2 dash, 5 circle
 \startitemize[2]
 \item Weather patterns in NA and Europe
 \item Interrelationships between ocean, atmosphere, Arctic systems
 \item Meteorologists \& general public
 \stopitemize

\stopslide

\page
\title{Review of Literature}

\startslide[title={\em History of Sea Ice in the Arctic}]
 \cite{Polyak2010}
 \startitemize[2]
   \item Seasonal levels
   \item Shrinking
     \startitemize[5]
     \item Warming
     \item Extended summer melt season
     \item Changing atmospheric patterns (clouds \& heating)
     \stopitemize
 \stopitemize
\stopslide

\startslide[title={\em An Arctic Wild Card in the Weather}]
  % http://en.wikipedia.org/wiki/File:Photo_of_second_North_American_blizzard_in_South_Riding,_Virginia.jpg
  \midaligned{\externalfigure[snowmageddon.jpg][height=300pt]}

  \page

 \cite{Greene2012}
 \startitemize[2]
   \item Arctic Oscillation
   \item Exposed ocean absorbs heat—\quotation{ice-albedo feedback mechanism}
   \item Heat released via evaporation…
   \item Increased pressure, moisture = weakened polar vortex/jet stream
   \item Polar vortex contains cold air
   \item Jet stream controls persistence in middle latitudes
 \stopitemize
\stopslide

\startslide[title={\em Bifurcations leading to summer Arctic sea ice loss}]
 \cite{Abbot2011}
 \startitemize[2]
   \item Bifurcation = qualitative change in system
   \item Mathematical model can’t make {\em specific} predictions…
   \item No saddle node bifurcations found, but Hopf unstudied…
 \stopitemize

 \page
 \dontleavehmode\blank[3cm]
 \startformula\startalign
   \NC \frac{dE}{dt} = \NC (1 - α(E))F_s(t) - A(E) - BT(E, t) \NR
   \NC \NC  + ΔA_{ghg} + (-E\mathcal{vR}(-E)) \NR
 \stopalign\stopformula
\stopslide

\startslide[title={\em The Slow Passage Through a Hopf Bifurcation}]
 \cite{Baer1989}
 \startitemize[2]
 \item Slow passage = continuous parameter that \quotation{slowly} changes
 \item Can lead to problems in computer calculations
 \item Solution: more precision \& {\bf awareness} of this problem
 \stopitemize

 \page

 % http://upload.wikimedia.org/wikipedia/commons/thumb/7/7d/LogisticMap_BifurcationDiagram.png/800px-LogisticMap_BifurcationDiagram.png
 \midaligned{\externalfigure[bifurcationdiagram][height=300pt]}
 \startformula
   x_{n+1} = r \, x_n(1-x_n)
 \stopformula

 \page
 \midaligned{\externalfigure[baerdiagram][height=300pt]}
\stopslide

\startslide[title={Current State of Knowledge}]
 \startitemize[2]
   \item Arctic ice melting = changing weather patterns…
   \item …and the ice is melting
   \item Already have had many severe winter storms
   \item Models don’t predict a specific type of qualitative change in the
     future
 \stopitemize
\stopslide

\startslide[title={Shadow Area}]
 \startitemize[2]
 \item Jargon of two fields: climate modeling and bifurcation theory
 \item Details of these weather systems
 \item How to apply conclusions from a naïve model
 \item Better understand the mathematics
 \stopitemize
\stopslide

\startslide[title={Outstanding Questions}]
 \startitemize[2]
   \item Are other types of bifurcations possible?
   \item Bifurcation generally means a sudden change
   \item How accurate can we make these models?
   \item How well do the models’ predictions relate to real weather?
 \stopitemize
\stopslide

\startslide[title={Research Question}]
{\ssa
  How does the level of greenhouse gas as a slow parameter (one that
  continuously varies at a slow rate) affect the behavior of an Arctic sea
  ice model, and how can those conclusions be applied to a general climate
  model?
}
\stopslide

\startslide[title={Proposed Procedure}]
 \startitemize[n]
   \item Implement model
   \item Run model with varying parameter
   \item Generate diagrams
   \item Analyze
 \stopitemize

 \page

 \setupinterlinespace[2ex]
 \setuptyping[style=\ttx,space=fixed]
 \starttyping

def runge_kutta(f, t0, y0, h, steps):
    t = t0
    y = y0
    for n in range(steps):
        k1 = f(t, y)
        k2 = f(t + h / 2, y + k1 * (h / 2))
        k3 = f(t + h / 2, y + k2 * (h / 2))
        k4 = f(t + h, y + k3 * h)

        t += h
        y += (h / 6) * (k1 + 2 * k2 + 2 * k3 + k4)

    return t, y

\stoptyping
 \setupinterlinespace[2.8ex]

 \page

 Materials needed:
 \startitemize[2,nowhite]
 \item Computer
 \item Time
 \stopitemize

 Assistance: Dr.\ Baer, to help verify calculations
\stopslide

\page[yes]

\BibFile{summary.bib}
\ReadList{bibliography}

\stoptext
%%% Local Variables:
%%% mode: context
%%% TeX-master: t
%%% End:
