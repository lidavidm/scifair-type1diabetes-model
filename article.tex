\usemodule[simplefonts][size=12pt]
\setuppapersize[letter]
\setuplayout[width=6.5in,height=10.5in,topspace=0.5in,backspace=1in,
  header=0.5in,footer=0.5in,bottomspace=0.5in]
\definefontfeature[default][default][mode=node, kern=yes, liga=yes, tlig=yes,trep=yes,protrusion=quality,expansion=quality]
\setuppagenumbering[state=off]
\setupwhitespace[medium]
\setuphead[title][style={\tfb}, after={}]
\setmainfont[Calluna]
\setupinterlinespace[3.5ex]
\starttext

\title{Arctic ice melt sets stage for severe winters, scientists say}

http://news.cornell.edu/stories/2012/06/arctic-ice-melt-sets-stage-cold-weather

Summary:

Contrary to most people’s expectations, the melting of Arctic ice due to
climate change will affect inland climate as much as coastal weather. Warmer
temperatures increase the amount of ice that melts during the summer,
exposing more of the ocean, which stores and releases heat energy,
especially during the fall, which decreases the temperature and pressure
gradients in the atmosphere. This affects weather patterns and allows cold
air to affect lower latitudes than before. Normally this occurs in a process
known as the Arctic Oscillation, a variation in the climate that normally
alternates between conditions that would allow or not allow Arctic air
south; melting the Arctic ice affects this pattern.

The article cited as an example a recent winter where temperatures reached
-22 Fahrenheit. This concerns Americans (perhaps not Arizonans) because of
such an increased severity in winters; winter storms can damage
infrastructure such as roads and power lines, leading to loss of life and
significant repair expenses. Thus, being able to model and analyze the
melting of Arctic ice would allow scientists to better determine the impact
on weather patterns. An opportunity for research exists in modeling the
melting of Arctic ice to determine how much can be expected to melt in
coming years and to determine when all will begin melting in the summer, or
perhaps in integrating Arctic ice into weather models.

\stoptext

%%% Local Variables:
%%% mode: context
%%% TeX-master: t
%%% End:
