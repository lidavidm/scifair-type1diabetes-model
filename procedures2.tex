\environment env_essay
\usemodule[tabl-nte]
\input bibliography
\setmonofont[Consolas][scale=0.9]
\indenting[no]
\setupitemgroup[itemize][indenting=no]
\setupwhitespace[small]
\def\mydots{\leavevmode\xleaders\hbox to 0.2em{\hfil.\hfil}\hfill\kern0pt}

\setuphead[title][style=\tfb,after={},before={}]
\setuphead[section][number=no,style=\tfa,before={\blank[0.2cm]},after={\blank[0.1cm]}]
\setuphead[subsection][number=no,style=\tf\em,before={},after={}]
\setuplabeltext[en][appendix=Appendix~]
\definehead[appendix][chapter]
\def\Appendix#1{Appendix #1.}
\setuphead[appendix][before={}, after={}, page=no, numbercommand=\Appendix,style={\tfa}, number=yes,conversion={A}]

\starttext

\title{Proposed Procedures: Continuous Analysis of a Model for T Cells in
  Autoimmune Diabetes}

David Li

\startsection[title={Materials}]

  Estimated Total Cost\mydots \$0

  \startitemize[n,joinedup]

    \startitem Computer (x86-64 architecture running a Linux distribution; no
      need to purchase)\mydots \$0

      The exact specifications do not matter, nor should the operating
      system, as the software used runs on all platforms. A relatively
      recent (past 5 years or so) computer will absolutely suffice.
    \stopitem

    \startitem Computer Software\mydots \$0

      \startitemize[1]
      \item Python 3.3.2 with {\tt mpmath} (\hyphenatedurl{http://www.python.org},
        \hyphenatedurl{http://www.mpmath.org}). This will be used to calculate the
        data for the plots.
      \item XPP AUTO 7.0 (http://www.math.pitt.edu/\textasciitilde
        bard/xpp/xpp.html). This will be used to calculate the data for the
        bifurcation diagrams.
      \item \CONTEXT \ 2013.05.28
        (http://wiki.contextgarden.net)\footnote{Inkscape may also be
        required to be able to generate SVG diagrams.}. This will be used to
        generate the plots and bifurcation diagrams (explained below).
      \item Gnuplot 4.6.4 (http://www.gnuplot.info), also used to generate
        plots.
      \stopitemize

    \stopitem

  \stopitemize

\stopsection

\startsection[title={High-Level Procedural Overview}]

  \startsubsection[title={Background Details}]
    {\em See the \about[app:dp] appendix on \at{page}[app:dp] for more
      details.}

    This experiment centers on the bifurcation analysis of the model
    designed by Mahaffy and Keshet as described at \cite{Mahaffy2007}.
    Their model describes the level of T cells reactive to a peptide found
    on pancreatic beta cells. These immune system cells destroy the
    insulin-producing cells, leading to type 1 diabetes. The level of T
    cells in the weeks before the onset of diabetes (the appearance of
    symptoms) oscillates, as demonstrated by previous experiments
    [\cite{Trudeau2003} cited in \cite{Mahaffy2007}]. One of the parameters
    in the model\footnote{The original paper contains multiple variants of
      the model; we are focusing on the {\em scaled reduced model} here.} is
    the level of beta cells, which slowly decreases over time as the disease
    progresses; after a certain point, symptoms begin to appear. The
    original analysis centered on indirectly varying the level of beta cells
    (via another parameter, the {\em peptide clearance rate} $δₚ$) to look
    for {\em bifurcations}, or qualitative changes in the model’s
    behavior. In particular, the authors considered the long-term behavior
    of the model system at multiple constant values of the parameter
    \cite{Mahaffy2007}.

    This experiment applies the research of Dr.\ Baer, who found that when
    such parameters are slowly and continuously varied, the qualitative
    nature of the system differs from when the parameter is statically set
    as described before \cite{Baer1989}. This change more accurately
    reflects what actually occurs, as the original paper explicitly states
    the parameter should continuously slowly fall\cite{Mahaffy2007}. Thus,
    this experiment will help scientists and medical professionals better
    understand and apply the theoretical results of Mahaffy and
    Edelstein-Keshet’s work as well as the experimental results they cited.

    To summarize: for the original {\em static} analysis, the authors re-ran
    the model multiple times, each time setting the parameter to a fixed
    value. For the {\em continuous} analysis conducted here, the parameter
    starts at a given value and then continuously decreases over time.
  \stopsubsection

  \startsubsection[title={Procedural Overview}]
    XPP AUTO, the same software used in the original experiment, is used to
    compute the solution of

    \startformula
      p = \frac{a_{14}}{a_{15}}EB \;\;\;\;\;
      f₁ = \frac{p^{a_1}}{{a_2}^{a_1}+p^{a_1}} \;\;\;\;\;
      f₂ = \frac{{a_4}{a_5}^{a_3}}{{a_5}^{a_3}+p^{a_3}}
    \stopformula
    \startformula\startalign
      \frac{dA}{dt} \NC = \NC(a₆+a₇M)f₁(p) - a₈A - a₉A² \NR
      \frac{dM}{dt} \NC = \NC a_{10}f₂(p)A-f₁(p)a₇a_{16}M - a_{11}M \NR
      \frac{dE}{dt} \NC = \NC a_{12}(1-f₂(p))A-a_{13}E \NR
    \stopalign \stopformula

    with
    $ a_{15} = δₚ = \mfunction{constant} $
    for the static analysis or
    $ \dfrac{da₁₅}{dt} = \dfrac{dδₚ}{dt} = \mfunction{rate~of~change} $
    for the continuous analysis\footnote{The rate of change depends on the
      parameter range in consideration (see the appendix).}. Here, $A$
    represents the level of activated T cells, $B$ the level of pancreatic
    beta cells, $E$ the effector T cells, $M$ the memory T cells, and $p$
    the peptide level. (The fluctuation of activated T cells matters the
    most for the onset of diabetes.) In the model, pancreatic beta cells
    that undergo apoptosis (cell death) generate peptide; in lymph nodes, T
    cells become activated and \quotation{recognize} this particular
    peptide. These cells then become either effector cells, which destroy
    more pancreatic cells, thus producing more peptide, or memory cells,
    which encounter the peptide and trigger an immune response.

    Because $B$ and $δₚ$ are constant, an increase in the latter is
    equivalent to a decrease in the former (and vice-versa), so the latter
    was increased in Mahaffy’s analysis, which is equivalent to a decrease
    in beta cell levels \cite{Mahaffy2007}.  Additionally, a 4\high{th}
    order Runge-Kutta ODE solver written in Python is also used to compute
    the numerical solution in the second case. Runge-Kutta is a standard
    algorithm for solving ODEs; it is used here because Dr.\ Baer’s work
    noted that the continuous analysis could lead to numerical precision
    errors, and the Python implementation allows for arbitrary-precision
    computations \cite{Baer1989}\cite{Mpmath2013}\cite{Gonsalves2009}.

    After simulation, the experiment will result in a set of bifurcation
    diagrams much like this one\footnote{These diagrams were generated by
      beta runs of the eventual experimental software.}:

    \input experiment/gnuplot
    \input experiment/plots

    \midaligned{
      \useGNUPLOTgraphic[bifurcation_xpp]
    }

    as well as model-over-time diagrams such as this one:

    \midaligned{
      \useGNUPLOTgraphic[time_xpp_12]
    }

    The bifurcation diagram shows the {\em maximum and minimum values of $A$
      over time for a trial run of the model with the particular parameter
      level indicated on the x-axis}. Thus, it shows at what {\em static}
    parameter levels diabetes can occur. For this experiment, a third
    diagram will be generated overlaying the model-over-time diagram for the
    {\em continuous} system onto the bifurcation diagram:

    \midaligned{
      \useGNUPLOTgraphic[bifurcation_combined_12]
    }

    Since the parameter will continuously and linearly vary over time, a
    correspondence between the points of the two graphs exists (i.e.\ for
    the model-vs-time graph, at a certain time, the parameter will be at a
    unique particular value in the X-range of the bifurcation diagram), and
    therefore the two modes of simulation can be compared – the goal of this
    experiment. (See the appendix for the implementation details of
    processing the data and generating the diagrams.)

    Additionally, this experiment will compare the static bifurcation
    diagrams of various other parameters when $δₚ$ is analyzed statically
    and continuously, in order to find out how these parameters interact
    with the continuous model versus the static model. $a_9 = ε$ controls
    the competition between T cells, and $a_5 = k₂$ represents the level of
    peptide at which $f₂$, the maximum number of activated T cells, is at
    half its maximum level.

  \stopsubsection

\stopsection

\page
\startappendix[title={Detailed Procedures}, reference=app:dp]

  {\em Note: familiarity with the Linux command line is assumed.}

\startitemize[n,joinedup]
\item Download the necessary experiment software and experiment files
  (location TBD).
\item Extract the experiment files to a directory.
\item Open a shell and {\tt cd} to the directory containing the experiment
  files.
\startitem To generate plots of the system over time:
  \startitemize[a]
  \item Run {\tt python3 rk4_ode.py} to generate the data via the Runge-Kutta
    method.
  \item Run {\tt ./time_diagrams_xppaut.sh} to generate the data via XPP
    AUTO.
  \item Run {\tt context time_diagrams.tex} to generate a PDF containing the
    plots.
  \stopitemize
\stopitem
\startitem Then, to generate the plots of the system vs the bifurcation
  parameters (the peptide clearance rate, $δₚ$; fraction of memory T cells
  produced, $a$; T cell competition parameter $ε$, and peptide level for ½
  max memory cells, $k₂$):
  \startitemize[a]
    \startitem[item:auto-steps] Run {\tt ../path_to_xpp/xppaut -runnow mk_static.ode} to
      launch XPP AUTO. Bring up AUTO and save the points for the complete
      bifurcation diagram in {\tt mk_static.dat}. If other parameters are to
      be analyzed, also save the diagrams for those cases.

      \startitemize[i]
      \item Press {\tt f a} to bring up AUTO.
      \item Press {\tt f l} and choose {\tt mk_static.ode.auto} to load the
        parameters for the bifurcation diagram.
      \item Press {\tt r s} to generate the diagram.
      \item Press {\tt g}, {\tt <TAB>} over to the bifurcation point
        (labeled {\tt HB} on the bottom), press {\tt <ENTER>} to select it,
        and then {\tt rp} to generate the branch.
      \item Repeat for each bifurcation point.
      \item Press {\tt f a} and save the points in {\tt mk_static.dat}.
      \item Close XPP AUTO ({\tt Control-C} the program in the terminal).
      \stopitemize
    \stopitem
  \item Repeat \in{step}[item:auto-steps] for {\tt mk_static_epsilon.ode}
    with {\tt mk_epsilon.ode.auto} and {\tt mk_static_k2.ode} with {\tt
      mk_k2.ode.auto}, and then for {\tt mk_continuous_epsilon.ode} and {\tt
      mk_continuous_k2.ode} with the corresponding {\tt .auto} files. Name
    the result file after the parameter and {\tt .ode} file (e.g. {\tt
      mk_continuous_epsilon.dat}, {\tt mk_static_k2.dat}).
  \item Run {\tt python3 bifurcation_diagrams.py} to process the plot data.
  \item Run {\tt context bifurcation_diagrams.tex} to generate a PDF
    containing the plots.
  \stopitemize
\stopitem
\stopitemize

\startsubsection[title={Parameters for XPP AUTO and Python}]
  The parameters in the following table are {\em not} parameters for the
  model, but rather for the programs used to generate the model data. All
  parameters not listed were left at default values.

  For the parameter $δₚ$:
  \startTABLE
    \setupTABLE[r][first][style=bold]
    \setupTABLE[c][1][width=.7\textwidth]
    \setupTABLE[c][2][width=.3\textwidth]
    \bTABLEhead
    \NC Parameter \NC Value \NC\NR
    \eTABLEhead

    \bTABLEbody
    \NC Integration time (the time the system will be run for) \NC 200 days \NC\NR
    \NC Integration step (RK4, both Python and XPP AUTO) \NC 0.05 days \NC\NR
    \NC {\tt mpmath} Decimal Precision ({\tt mpmath.mp.dps}) \NC 200 \NC\NR
    \NC AUTO Par 1 \NC {\tt a15} \NC\NR
    \NC AUTO Hi-Lo Y-axis \NC {\tt A} \NC\NR
    \NC AUTO Hi-Lo Main Parameter \NC {\tt a15} \NC\NR
    \NC AUTO Hi-Lo ranges \NC X: $\left[0, 18 \right]$ Y: $\left[0, 3\right]$ \NC\NR
    \NC AUTO Par Max \NC 18 \NC\NR
    \eTABLEbody

  \stopTABLE

  For the parameter $ε$ (assume other parameters are the same as before):
  \startTABLE
    \setupTABLE[c][1][width=.7\textwidth]
    \setupTABLE[c][2][width=.3\textwidth]

    \bTABLEbody
    \NC AUTO Par 1 \NC {\tt a9} \NC\NR
    \NC AUTO Hi-Lo Main Parameter \NC {\tt a9} \NC\NR
    \NC AUTO Hi-Lo ranges \NC X: $\left[0, 5 \right]$ Y: $\left[0, 3\right]$ \NC\NR
    \NC AUTO Par Max \NC 5 \NC\NR
    \NC AUTO Ds \NC -0.02 \NC\NR
    \eTABLEbody

  \stopTABLE

  For the parameter $k₂$ (assume other parameters are the same as before):
  \startTABLE
    \setupTABLE[c][1][width=.7\textwidth]
    \setupTABLE[c][2][width=.3\textwidth]

    \bTABLEbody
    \NC AUTO Par 1 \NC {\tt a5} \NC\NR
    \NC AUTO Hi-Lo Main Parameter \NC {\tt a5} \NC\NR
    \NC AUTO Hi-Lo ranges \NC X: $\left[0, 2.5 \right]$ Y: $\left[0, 3\right]$ \NC\NR
    \NC AUTO Par Max \NC 2 \NC\NR
    \NC AUTO Ds \NC -0.02 \NC\NR
    \eTABLEbody

  \stopTABLE
\stopsubsection

\startsubsection[title={Initial Conditions}]
For the model when processed over time:
\startformula
A = 0.5 \;\;\;\;\;
M = 0 \;\;\;\;\;
E = 1 \;\;\;\;\;
B = 1 \;\;\;\;\;
a_{15} = \mfunction{See~below}
\stopformula

For the model when generating bifurcation diagrams for the parameter
$a_{15}$:

\startformula
A = 0.01961524 \;\;\;\;\;
M = 2.802597\times10^{-45} \;\;\;\;\;
E = 0.006538414 \;\;\;\;\;
B = 1 \;\;\;\;\;
a_{15} = 0
\stopformula
\stopsubsection

\startsubsection[title={Parameter Ranges}]
  As the total time is 200 days, the parameter will be varied at a rate of $
  (δ_{p_2} - δ_{p_1})/200$ per day. These ranges were chosen based on a
  reading of Mahaffy’s paper \cite{Mahaffy2007}. Although the authors did
  not call out explicit parameter ranges, the ranges here represent either
  the domains of graphs given or are near bifurcation points they
  noted. Also, they tended to include a parameter value of $0$, which has
  been adjusted to $0.1$ here to avoid a division-by-zero error. XPP AUTO does
  not return an error when this occurs, but Python does, so the adjustment
  avoids conflicts and undocumented behaviors. This will not adversely
  affect the impact of this experiment; because the original analysis used
  static parameter values, the overall range of values does not matter as
  long as it includes the range used in the continuous analysis to be done
  here for comparison.

  \bTABLE
    \setupTABLE[r][first][style=bold]
    \setupTABLE[c][1][width=.7\textwidth]
    \setupTABLE[c][2][width=.3\textwidth]
    \bTABLEhead
    \bTR \bTD Parameter \eTD\bTD Ranges \eTD \eTR
    \eTABLEhead

    \bTABLEbody
    \bTR \bTD $δₚ$, the peptide clearance rate \eTD       \bTD 0.5 to 2.3 \eTD \eTR
    \bTR \bTD[nr=2] (continuous, others are static) \eTD  \bTD 0.1 to 2   \eTD \eTR
    \bTR                                                  \bTD 2 to 0.1   \eTD \eTR
    \bTR \bTD $ε$, the T cell competition parameter \eTD  \bTD 5 to 0     \eTD \eTR
    \bTR \bTD $k₂$, the half-maximal peptide level \eTD   \bTD 2 to 0     \eTD \eTR
    \eTABLEbody

  \eTABLE
\stopsubsection

\stopappendix

\startappendix[title={Bibliography}]

\placebibliography{summary.bib}

\stopappendix

\stoptext