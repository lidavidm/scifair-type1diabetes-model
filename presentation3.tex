\environment env_presentation_alt

\starttext

\title{Enhancing Theoretical Understanding of the Onset of Type 1 Diabetes}

\blank[1cm]
{\startalignment[center]

  David Li

  Dr.\ McKelvy

  8 Jan 2014

\stopalignment}

\page

\startslide[title={Research Question}]

  How does treating the level of beta cells as a {\em continuously} varying
  slow parameter affect the qualitative behavior of the scaled reduced
  immune model developed by Mahaffy and Edelstein-Keshet \cite{Mahaffy2007},
  and how can those findings be applied to understanding and predicting
  type 1 diabetes?

\stopslide

\startslide[title={Hypothesis}]

  If the model for the level of immune cells in the weeks before the onset
  of type 1 diabetes is analyzed with both a continuously varying and a
  static peptide clearance rate $δₚ$, then in the former analysis, the
  oscillations present in the original model will begin at a later time
  because research has shown this behavior is delayed in other models when
  analyzed with a continuously varying parameter.

\stopslide

\startslide[title={Materials}]
  \startitemize[1,nomargin]
  \item Computer
  \item Software:
  \stopitemize
  \setuptyping[before=,after=]

  \switchtobodyfont[20pt]
  \starttyping
% python --version
Python 3.3.3
% ./xppaut -version
XPPAUT Version 7.0
% python -c "import mpmath; print(mpmath.__version__)"
0.17
% gnuplot --version
gnuplot 4.6 patchlevel 4
% context --version
mtx-context     | ConTeXt Process Management 0.60…
mtx-context     | current version: 2014.01.03 00:40
\stoptyping
\switchtobodyfont[24pt]
\stopslide

\startslide[title={Procedure}]
\startitemize[n]
\item Using AUTO, compute the data for the bifurcation diagram.
\item For each parameter range:
  \startitemize[a]
  \item Run the model with AUTO
  \item Run the model with Python
  \stopitemize
\item Plot everything
\stopitemize
\page
\resetbg
\startitemize[1]
\item AUTO is a standard tool for bifurcation and ODE work in mathematical
  modeling
\item {\tt mpmath} is newer and not seen in the field; used to verify
  results
\item \CONTEXT{} and Gnuplot are for generating plots (\CONTEXT{} is a
  cousin of \LATEX{}, standard typesetting tool in the sciences)
\item Only two \quotation{trials}, but the experiment is
  deterministic—repetition unnecessary
\item Control group is the bifurcation diagram; comparisons can also be made
  to Mahaffy’s data and the original experiment
\item Outside factors: round-off error (reason for Python)
\stopitemize

\stopslide

\startslide[title={Data}]

\input gnuplot_large
\input experiment/plots

\def\showplot#1{
  \switchtobodyfont[18pt]
  \startcolor[black]
    \framed[background=color,backgroundcolor=white]{
      \useGNUPLOTgraphic[#1]
    }
  \stopcolor
  \switchtobodyfont[24pt,mainface]
}

\showplot{bifurcation_xpp}

\page \resetbg

\showplot{bifurcation_combined_12}
Oscillations start {\em before} the bifurcation point, but don’t become
noticeable until after

\showplot{bifurcation_combined_12_long}
On a less realistic time scale (1000 days vs 200), the oscillations start
much later. If the beta cell decline can somehow be slowed…

\showplot{bifurcation_combined_h2}
Starting from a condition with few beta cells/a high clearance rate shows no
oscillations. Biologically, the disease has already set in…trying to lower
\math{\delta_p} doesn’t help

\stopslide

\startslide[title={Potential Impact}]
\startitemize[1]
\item Two interpretations: \math{\delta_p} vs \math{B} (equivalent effect)
\item Sufficiently lowering \math{\delta_p} could delay the onset
\item Increasing \math{B} or decreasing \math{\delta_p} too late does
  nothing, of course
\item Impact: speed of increase affects onset time; may contribute to
  explanation of individual variance
\item Applications: look for treatments that can manipulate these variables,
  tests that can monitor them…
\item Future work: address Mahaffy’s concerns with his model to make it more
  accurate
\item Having the original experimental data for comparison would be helpful
\stopitemize
\stopslide

\startslide[title={Further Analysis}]
\startitemize[1]
\item Dr.\ Baer pointed me to the WKB method, used to determine exact point
  at which oscillations begin in such studies
\item Also suggested the idea of a \math{\mfunction{Re}(\lambda)} vs
  \math{\delta_p} graph, another way to tell when oscillations begin
  (contained in his paper)
\item One issue with the latter…
\stopitemize

\page\resetbg
\showplot{eigenvalues} \math{\mfunction{Re}(\lambda)} never crosses the
axis!
\stopslide

\BibFile{summary.bib}

\stoptext
%%% Local Variables:
%%% mode: context
%%% TeX-master: t
%%% End:
