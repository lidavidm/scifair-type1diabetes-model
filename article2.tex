\usemodule[simplefonts][size=10pt]
\setuppapersize[letter]
\setuplayout[width=6.5in,height=10.5in,topspace=0.5in,backspace=1in,
  header=0.5in,footer=0.5in,bottomspace=0.5in]
\definefontfeature[default][default][mode=node, kern=yes, liga=yes, onum=yes, protrusion=quality,expansion=quality]
\setuppagenumbering[state=off]
\setupwhitespace[medium]
\setuphead[title][style={\tfb}, after={}, before={}]
\setmainfont[Calluna]
\setmathfont[euler]
\setupinterlinespace[3ex]
\starttext

\title{Numerical modeling of sea ice in the climate system (excerpts)}

http://www.atmos.washington.edu/~bitz/Bitz_draftchapter.pdf

1.2 Sea Ice Physics in Leading Climate Models

State of the art climate models today treat the jumble of sea ice floes as a
continuum. Thus sea ice is generally described in terms of a distribution of
sea ice thicknesses at the subgridscale. The ice motion is also considered
for a continuum, rather than for individual floes.  With this brief overview,
a global-scale sea ice model can be conveniently developed from four
governing equations.

Four Governing Equations

The best sea ice components in global climate models today explicitly
compute the icethickness distribution (see Table 2). Among these models, the
formulation of the sea ice model begins with the ITD equation. The ITD is a
probability density function (pdf), usually written $g(h)$, that describes
the probability that the ice cover in particular region has thickness $h$. A
cruder alternative is to model the mean thickness of the pdf and the total
ice concentration.  In a sea ice model, the ITD describes the pdf of a grid
cell and thus it is sometimes called a subgrid-scale parameterization. A
parameterization typically represents processes that are too small-scale or
complex to be represented explicitly. For example, deformation is
parameterized with a set of rules that select the portion of the ITD that
will deform and then redistribute it within the ITD. In contrast, ice growth
and melt alter the ITD in a way that is computed from first principles. Hence
the ITD actually includes both parameterized and explicit physics…

There are two parts to deformation: a rate of opening (creating open water)
and closing (closing open water and/or deforming and redistributing the
ice), which depend on u, and a redistribution process (or ridging mode),
which depends on $g(h)$. The opening and closing rates depend on the
convergence and/or shear in the ice motion field. It may not be obvious that
shear would cause deformation. Imagine that the ice pack is composed of
pieces with jagged edges. When shearing, the jagged edges can catch on one
another and cause deformation, which converts kinetic energy into potential
energy from piling up ice, or shearing can cause frictional loss of energy
and no deformation. Thus the closing rates also depend on assumptions made
about frictional losses, see e.g., (Flato and Hibler, 1992; Bitz et al.,
2001).

For the redistribution process, some portion of the ITD is identified as
potentially able to “participate” in redistribution (see Fig. 1). A typical
rule assumes only the thinnest 15\% of the ITD participates. If the open
water fraction exceeds 15\%, then no redistribution takes place, and instead
the open water closes under convergence and nothing happens under shear.
This participation function is weighted according to its thickness, so that
the thinnest ice is most likely to deform. Another rule is needed to
redistribute the ice that ridges. Originally, Thorndike et al. (1975)
proposed that ice that ridges would end-up five times thicker than its
starting thickness. Other more complex redistribution schemes have been used
since then (e.g., Hibler, 1980; Lipscomb et al., 2007).

Ice growth or melt causes g(h) to shift along its x-axis, or thickness
space. This process is illustrated in Fig. 2. The growth/melt rate depends
on thickness, so $g(h)$ becomes distorted in the process.

The second governing equation is conservation of momentum…Conservation of
enthalpy $E$ (the energy required to melt sea ice) is the third governing
equation…

Climate model resolution is nearing the large floe-scale, where sea ice can
no longer be considered a continuum. I expect some climate models will adopt
non-continuum sea ice dynamics in the 5-10 yr timeframe to break the
floe-scale resolution limit. In the next 5 years, I expect the major new
physics to be added to sea ice models will be explicit melt ponds; better
radiative transfer and snow morphology; and primitive salinity, fluid
transport, and biogeochemistry. The greatest computational increase could be
to advect many new variables if it weren’t for schemes such as incremental
remapping which can efficiently transport large numbers of sea ice state
variables (Lipscomb and Hunke, 2004). Computational resources will not be
the limiting factor to implementing these new features. Instead, expertise
in sea ice and polar climate at the modeling centers and errors in the other
climate model components will limit advances. Learning how sea ice models in
today’s global climate models work is the first step to meet this certain
need.

\title{Studying : Modeling}

http://nsidc.org/cryosphere/seaice/study/modeling.html

Sea ice models have a more specific focus than General Circulation Models
(GCMs), which simulate the climate of the entire world. You may have read
about GCMs in the newspaper or a science magazine. They are often used to
investigate the effects of global warming caused by increasing carbon
dioxide, including the studies used for the Intergovernmental Panel on
Climate Change (IPCC) (visit the IPCC Web site). Until recently, sea ice has
been included in a rudimentary manner in GCMs. Because GCMs have shown that
the the polar regions are particularly sensitive to small changes in
climate, newer GCMs have improved sea ice models considerably.

Sea ice models provide valuable information on how sea ice evolves and how
it will be affected by changing climate.

Numerical sea ice models are primarily used to study the important processes
involved in the evolution of sea ice and are used for long-term climate
studies. However, models are also used to provide short-term operational
forecasts (one to five days) for ocean vessels in sea ice-covered regions,
as well as seasonal forecasts (one to three months) to aid in planning. The
U.S. Navy, for example, runs an operational sea ice model, the Polar Ice
Prediction System (PIPS), to provide short-term forecasts of Arctic sea ice
(visit the PIPS Web site).

The equations in a sea ice model describe the relevant dynamics and
thermodynamics that influence the evolution of sea ice. The dynamics
equations take into account winds, currents, and other forces that influence
sea ice motion. The thermodynamics equations take into account air and ocean
temperatures, albedo, and other forces that influence the growth and melt of
sea ice.

The evolution of sea ice is influenced by the ocean below and the atmosphere
above. These influences are represented in models as boundary conditions, or
conditions outside of the model, at the top and bottom of the sea
ice. Boundary conditions are more specifically referred to as forcings,
because they force the sea ice to change in a certain way, based on
influences from the ocean and atmosphere. An example is that air
temperatures above freezing cause the sea ice to melt, whereas air
temperatures below freezing cause the sea ice to grow.

The processes involved in sea ice evolution are too complex to be exactly
described by mathematical equations within models. These equations are
approximations that capture only the processes necessary for the model
application. Models carry assumptions, and predictions from the models have
inherent uncertainties. In theory, a more complex model provides a more
realistic simulation. However, the accuracy of the model is limited
internally by the model physics, and externally by the boundary conditions,
or forcing. As computers become more powerful and sea ice models are
improved, simulations from models will have fewer uncertainties.

Sea ice models are often combined with ocean or atmospheric models. These
are called coupled models, because rather than specifying the ocean or
atmosphere as forcings, the sea ice, ocean, or atmosphere interact with each
other and all the components evolve together. Coupled models can include sea
ice and ocean, sea ice and atmosphere, or all three.

How sea ice models work

Numerical models represent sea ice in cells. A cell is the smallest discrete
area that can be described by the model. Within each cell, there is no
variability in space of any of the properties. This property of model cells
is similar to an image captured by a digital camera, where each image pixel
is only one color. Each cell has a finite area, for example, 10 square
kilometers (3.9 square miles) and describes several dynamic and
thermodynamic properties, such as thickness, temperature, and velocity. The
entire group of cells is referred to as a domain. As the number of grid
cells increases in a model domain, so does the spatial resolution. Having a
large number of grid cells describes sea ice conditions on smaller
scales--tens of kilometers, versus hundreds). However, the cost of storing
and computing the model increases substantially.

The properties of each grid cell change as the model is run forward in time,
based on the model equations and boundary conditions.

After a model is run, the output is essentially a set of relevant sea ice
variables, such as concentration, extent, and thickness, over time at each
grid cell. Model outputs can be evaluated using sea ice observations (such
as satellites and buoys) to determine how accurate the model is.

\title{Arctic ice melt sets stage for severe winters, scientists say}

http://news.cornell.edu/stories/2012/06/arctic-ice-melt-sets-stage-cold-weather

The dramatic melt-off of Arctic sea ice due to climate change is hitting
closer to home than millions of Americans might think. That's because
melting Arctic sea ice can trigger a domino effect leading to increased odds
of severe winter weather outbreaks in the Northern Hemisphere's middle
latitudes -- think the "Snowmageddon" storm that hamstrung Washington, D.C.,
during February 2010.

Cornell's Charles H. Greene, professor of earth and atmospheric sciences,
and Bruce C. Monger, senior research associate in the same department,
detail this phenomenon in a paper published in the June issue of the journal
Oceanography.

"Everyone thinks of Arctic climate change as this remote phenomenon that has
little effect on our everyday lives," Greene said. "But what goes on in the
Arctic remotely forces our weather patterns here."

A warmer Earth increases the melting of sea ice during summer, exposing more
dark ocean water to incoming sunlight. This causes increased absorption of
solar radiation and excess summertime heating of the ocean -- further
accelerating the ice melt. The excess heat is released to the atmosphere,
especially during the autumn, decreasing the temperature and atmospheric
pressure gradients between the Arctic and middle latitudes.

A diminished latitudinal pressure gradient is linked to a weakening of the
winds associated with the polar vortex and jet stream. Since the polar
vortex normally retains the cold Arctic air masses up above the Arctic
Circle, its weakening allows the cold air to invade lower latitudes.

The recent observations present a new twist to the Arctic Oscillation (AO)
-- a natural pattern of climate variability in the Northern
Hemisphere. Before humans began warming the planet, the Arctic's climate
system naturally oscillated between conditions favorable and those
unfavorable for invasions of cold Arctic air.

Greene says, "What's happening now is that we are changing the climate
system, especially in the Arctic, and that's increasing the odds for the
negative AO conditions that favor cold air invasions and severe winter
weather outbreaks.

"It's something to think about given our recent history," Greene
continued. This past winter, an extended cold snap descended on central and
eastern Europe in mid-January, with temperatures approaching -22 Fahrenheit
and snowdrifts reaching rooftops. And, of course, there were the record
snowstorms fresh in the memories of residents from several eastern
U.S. cities, such as Washington, New York and Philadelphia, as well as many
other parts of the Eastern Seaboard during the previous two years.

But wait -- Greene and Monger's paper is being published just after one of
the warmest winters in the eastern U.S. on record. How does that relate?

"It's a great demonstration of the complexities of our climate system and
how they influence our regional weather patterns," Greene said. In any
particular region, many factors can have an influence, including the El
Nino/La Nina cycle. This winter, La Nina in the Pacific shifted undulations
in the jet stream so that while many parts of the Northern Hemisphere were
hit by the severe winter weather patterns expected during a bout of negative
AO conditions, much of the eastern United States basked in the warm tropical
air that swung north with the jet stream.

"It turns out that while the eastern U.S. missed out on the cold and snow
this winter, and experienced record-breaking warmth during March, many other
parts of the Northern Hemisphere were not so fortunate," Greene said. Europe
and Alaska experienced record-breaking winter storms, and the global average
temperature during March 2012 was cooler than any other March since 1999.

"A lot of times people say, 'Wait a second, which is it going to be -- more
snow or more warming?' Well, it depends on a lot of factors, and I guess
this was a really good winter demonstrating that," Greene said. "What we can
expect, however, is the Arctic wildcard stacking the deck in favor of more
severe winter outbreaks in the future."

\page

\tfa
\setupinterlinespace[3.15ex]

\title{Summary}

The three articles discuss the state of Arctic sea ice melting and
modeling. Stanford’s article, while light on technical details, provides a
high-level overview of the problem and its implications: melting Arctic ice,
while not likely to raise sea levels (as the ice is floating in the ocean),
will affect weather patterns within the United States. Because the lack of
ice exposes more seawater, which stores and releases thermal energy, this
affects the Arctic Oscillation, a weather pattern that controls the flow of
cold Arctic air southward. However, the scientists in the article admit that
even with modeling of Arctic ice, they cannot predict its effect on weather
patterns too accurately due to the complex interactions between various
systems, though the general qualitative effect is known.

The NSIDC article gives a high-level overview of the computational models
used to study the melting of ice. Generally, such models are integrated into
more general GCMs (global climate models) rather than used standalone. The
models take into account oceanic and atmospheric influences on the ice and
use simplified physics to model how the ice melts and forms. Generally, the
algorithm divides the ice into cells, the smallest discrete area that the
model attempts to describe. Each cell has properties such as thickness and
temperature; more cells leads to better accuracy but also more computational
cost. Cells can range from 10 square kilometers on a side to hundreds on a
side. Then, the model is run and modifies those properties over time. Again,
the authors concede that such models make many assumptions and note that
because ice, ocean, and atmosphere are interdependent, integrated models can
make better predictions. Generally satellites, buoys, and other measurements
are used to verify the model predictions, but such measurements are limited.

The Washington article describes some of the lower-level mathematical
concepts in such models. In particular, models use a probability density
function that represents the likelihood that the ice in a region has a
particular thickness. The author describes several potential improvements,
particularly in integrating more of the relevant physics as well as improved
algorithms that could potentially improve computational time. Such advances
are candidates for research, though implementing such a model efficiently
may not be feasible. Again, this author also notes the need for integration
with other climate models in order to receive accurate predictions.

As for research topics, some models that are simpler, and in particular, use
differential equations directly are potential candidates for bifurcation
analysis, particularly if the model has a parameter that can be changed over
time (e.g.\ quantity of greenhouse gases in the atmosphere). The models
described here are more numerical in nature and are too complicated for this
analysis, but other models would be amenable to analysis. Varying such a
parameter would amount to a crude integration with other climate models,
which all articles note is a necessity for accurate simulation.

%%% Local Variables:
%%% mode: context
%%% TeX-master: t
%%% End:
