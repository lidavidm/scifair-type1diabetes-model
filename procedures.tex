\environment env_essay
\usemodule[tabl-nte]
\input bibliography
\setmonofont[Consolas][scale=0.9]
\indenting[no]
\setupitemgroup[itemize][indenting=no]
\setupwhitespace[small]
\def\mydots{\leavevmode\xleaders\hbox to 0.2em{\hfil.\hfil}\hfill\kern0pt}

\setuphead[title][style=\tfb,after={},before={}]
\setuphead[section][number=no,style=\tfa,before={\blank[0.5cm]}]
\setuplabeltext[en][appendix=Appendix~]
\definehead[appendix][chapter]
\def\Appendix#1{Appendix #1.}
\setuphead[appendix][before={}, after={}, page=no, numbercommand=\Appendix,style={\tfa}, number=yes,conversion={A}]

\starttext

\title{Proposed Procedures: Continuous Analysis of a Model for T Cells in
  Autoimmune Diabetes}

David Li

\startsection[title={Materials}]

Estimated Total Cost\mydots \$0

\startitemize[n,joinedup]

\startitem Computer (x86-64 architecture running a Linux distribution; no
  need to purchase)\mydots \$0

For reference, the computer used in this experiment has the following
specifications:

\startitemize[1]
\item Processor: Intel Core i5-3230M
\item Memory: 4GB
\item Linux distribution: Arch Linux
\stopitemize
\stopitem

\startitem Computer Software\mydots \$0

\startitemize[1]
\item Python 3.3.2 with {\tt mpmath} (arbitrary-precision arithmetic)
  (\hyphenatedurl{http://www.python.org}, \hyphenatedurl{http://www.mpmath.org})
\item XPP AUTO 7.0 (bifurcation analysis)
  (http://www.math.pitt.edu/\textasciitilde bard/xpp/xpp.html)
\item \CONTEXT \ 2013.05.28 (typesetting) (http://wiki.contextgarden.net)
\item Gnuplot (plotting)
\stopitemize

\stopitem

\stopitemize

\stopsection

\startsection[title={Procedures}]

  {\em Note: familiarity with the Linux command line is assumed, though the
    software used should run on Windows. Also note: {\tt diagrams.py}, {\tt
      time_diagrams.tex}, and {\tt bifurcation_diagrams.tex} do not yet
    exist. Further automation scripts may eventually be created, thus the
    procedures may differ.}

\startitemize[n,joinedup]
\item Download the necessary experiment software and experiment files
  (location TBD).
\item Extract the experiment files to a directory.
\item Open a shell and {\tt cd} to the directory containing the experiment
  files.
\startitem To generate plots of the system over time:
  \startitemize[a]
  \item Run {\tt python rk4_ode.py} to generate the data via the Runge-Kutta
    method.
  \item Run {\tt ../path_to_xpp/xppaut -silent mk.ode} to generate the data
    via XPP AUTO.
  \item Run {\tt python diagrams.py time} to generate the \TEX{} plot files.
  \item Run {\tt context time_diagrams.tex} to generate a PDF containing the
    plots.
  \stopitemize
\stopitem
\startitem Then, to generate the plots of the system vs the bifurcation
  parameter (the peptide clearance rate, $δₚ$):
  \startitemize[a]
  \startitem Run {\tt ../path_to_xpp/xppaut -runnow mk_static.ode} to launch XPP
    AUTO.
    \startitemize[i]
    \item Press {\tt f a} to bring up AUTO.
    \item Press {\tt f l} and choose {\tt mk_static.ode.auto} to load the
      parameters for the bifurcation diagram.
    \item Press {\tt r s} to generate the diagram.
    \item Press {\tt g <TAB> <ENTER> r p} to generate the branches for the
      first bifurcation.
    \item Press {\tt g <TAB> <TAB> <ENTER> r p} to generate the branches for
      the second bifurcation.
    \item Press {\tt f w} and save the points in {\tt mk_static.dat}.
    \item Close XPP AUTO ({\tt Control-C} the program in the terminal).
    \stopitemize
  \stopitem
  \item Run {\tt python diagrams.py bifurcation} to generate the \TEX{} plot
    files.
  \item Run {\tt context bifurcation_diagrams.tex} to generate a PDF
    containing the plots.
  \stopitemize
\stopitem
\stopitemize

Parameter Values (for the programs, not the system of ODEs):
\blank[line]

\setupTABLE[r][first][style=bold]
\startlocalfootnotes
\startTABLE
\setupTABLE[c][1][width=.7\textwidth]
\setupTABLE[c][2][width=.3\textwidth]
\bTABLEhead
\NC Parameter \NC Value \NC\NR
\eTABLEhead

\bTABLEbody
\NC Integration time (the time the system will be run for) \NC 200 days \NC\NR
\NC Integration step (RK4)\footnote[foot:int-step]{RK4 and XPP AUTO use
  different integration algorithms, thus a different step size should not
  matter.} (the time interval used) \NC 0.01 days \NC\NR
\NC Integration step (XPP AUTO)\note[foot:int-step] \NC 0.05 days \NC\NR
\NC AUTO Par 1 \NC {\tt a15} \NC\NR
\NC AUTO Hi-Lo Y-axis \NC {\tt A} \NC\NR
\NC AUTO Hi-Lo Main Parameter \NC {\tt a15} \NC\NR
\NC AUTO Hi-Lo X range \NC $\left[0, 18 \right]$ \NC\NR
\NC AUTO Hi-Lo Y range \NC $\left[0, 3\right]$ \NC\NR
\NC AUTO Par Max\footnote{All parameters for AUTO not listed were left at
  their default values.} \NC 18 \NC\NR
\eTABLEbody

\stopTABLE

Initial Conditions
\blank[line]

\startTABLE
\bTABLEhead
\NC Initial Conditions\footnote{$B$ was left at 1; all parameters had their
  default scaled values.} \NC Value of $A$ \NC Value of $M$ \NC Value of
$E$\NC Value of $a_{15}$\NC\NR
\eTABLEhead

\bTABLEbody
\NC \quotation{Typical} (diseased) state, used for the static bifurcation diagram \NC 0.5 \NC 0 \NC 1 \NC 1 \NC\NR
\NC Healthy state \#1 \NC 0 \NC 0 \NC 1 \NC 1 \NC\NR
\NC Healthy state \#2 \NC 0 \NC 0.5 \NC 1 \NC 0 \NC\NR
\eTABLEbody

\stopTABLE

Parameter Ranges (for the peptide clearance rate $δₚ$)

As the total time is 200 days, the parameter will be varied at a rate of $
\displaystyle\frac{δₚ₂ - δₚ₁}{200}$ per day.

\startitemize[1,joinedup]
\item 0 to 4
\item 0 to 18
\item 3 to 1.5
\stopitemize

\placelocalfootnotes
\stoplocalfootnotes

\stopsection

\startappendix[title={Program Files}]
  XPP AUTO ODE file (based on \cite{Mahaffy2007}). This file sets up the
  system of equations

\startformula \startalign
p \NC = \NC \frac{a_{14}}{a_{15}}EB \;\;\;\;\;
f₁ = \frac{pⁿ}{k₁ⁿ+pⁿ} \;\;\;\;\;
f₂ = \frac{ak₂ᵐ}{k₂ᵐ+pᵐ} \NR
\frac{dA}{dt} \NC = \NC(a₆+a₇M)f₁(p) - a₈A - a₉A² \NR
\frac{dM}{dt} \NC = \NC a_{10}f₂(p)A-f₁(p)a₇a_{16}M - a_{11}M \NR
\frac{dE}{dt} \NC = \NC a_{12}(1-f₂(p))A-a_{13}E \NR
\frac{dB}{dt} \NC = \NC -a_{17}EB \NR
\stopalign \stopformula
\setuptyping[escape=yes]
\starttyping

p = a14*E*B/a15
f1 = p^a1/(a2^a1+p^a1)
f2 = a4*a5^a3/(a5^a3+p^a3)
A' = f1*(a6+a7*M)-a8*A-a9*A^2
M'= a10*f2*A-f1*a16*a7*M-a11*M
E' = a12*(1-f2)*A-a13*E
# Disease-free initial conditions
# init A=0.5,M=0,E=1
# Diseased initial conditions
# init A=0.5,M=0.0,E=1

# Continuously decreasing analysis
# p is in the range 0-10 so a15 can be
# in the range 1 to infinity (p=0)
par B=1
a15'=/BTEX\em<RATE OF CHANGE OF PARAMETER>/ETEX
init a15=/BTEX\em<INITIAL PARAMETER VALUE>/ETEX
par a1=2,a2=2,a3=3,a4=0.7,a5=1,a6=0.02
par a7=20,a8=1.0,a9=1.0,a10=1,a11=0.01,a12=0.1
par a13=0.3,a14=50,a16=0.1,a17=0.14
@ dt=/BTEX\em<TIME STEP>/ETEX
@ total=/BTEX\em<FINAL TIME>/ETEX
@ xlo=0,xhi=200,ylo=0,yhi=4
@ NPLOT=4, XP=t, YP=A, XP2=t, YP2=M, XP3=t, YP3=E
done

\stoptyping

Python 4\high{th} order Runge-Kutta solver for a system of ODEs (based on
\cite{Gonsalves2009}).

\starttyping
from mpmath import mpf, mp

def vectorize(functions):
    def _vectorized(x, ys):
        return tuple(f(x, *ys) for f in functions)
    return _vectorized
def inc_vec(vec, inc):
    return [x + inc for x in vec]
def add_vec(*vecs):
    return [sum(nums) for nums in zip(*vecs)]
def mul_vec(vec, mul):
    return [x * mul for x in vec]
def rk4_system(y, x0, y0, h, steps):
    """
    y: list of ODE
    y0: vector of initial ys (tuple)
    Finds y(x0 + h * steps)
    """
    x1 = x0
    y1 = list(y0)
    xs = [x0]
    ys = [[y0[i]] for i,f in enumerate(y)]
    f = vectorize(y)
    for i in range(steps):
        k1 = f(x1, y1)
        k2 = f(x1 + (h / 2), add_vec(y1, mul_vec(k1, h / 2)))
        k3 = f(x1 + (h / 2), add_vec(y1, mul_vec(k2, h / 2)))
        k4 = f(x1 + h, add_vec(y1, mul_vec(k3, h)))
        y1 = add_vec(y1, mul_vec(add_vec(k1, mul_vec(k2, 2),
                     mul_vec(k3, 2), k4), h / 6))
        for i, val in enumerate(y1):
            ys[i].append(val)
        xs.append(x1)
        x1 += h
    return xs, ys


if __name__ == '__main__':
    /BTEX\em\# PARAMETER DEFINITIONS OMITTED (SEE PAPER FOR VALUES)/ETEX
    f_1 = lambda p: p**a1 / (a2**a1 + p**a1)
    f_2 = lambda p: a4 * a5**a3 / (a5**a3 + p**a3)
    p = lambda t, E, B, a15: a14 * E * B / a15
    y = (
        lambda t,A,M,E,B,a15:(a6+a7*M)*f_1(p(t,E,B,15))-a8*A-a9*A**2,#dA/dt
        lambda t,A,M,E,B,a15:a10*f_2(p(t,E,B,a15))*A-f_1(p(t,E,B,a15))*a7*a16*M-a11*M,#dM/dt
        lambda t,A,M,E,B,a15:a12*(1-f_2(p(t,E,B,a15)))*A-a13*E,#dE/dt
        lambda t,A,M,E,B,a15:-a17*E*B,#dB/dt
        lambda t,A,M,E,B,a15:0.1#da15/dt
    )
    result = rk4_system(y, mpf('0'), (/BTEX\em<INITIAL CONDITIONS>/ETEX), /BTEX\em<TIME STEP>/ETEX, /BTEX\em<STEPS>/ETEX)

    import pickle
    pickle.dump(result, open('rk4_ode.pickle', 'wb'))
\stoptyping

\stopappendix

\startappendix[title={Bibliography}]

\placebibliography{summary.bib}

\stopappendix

\stoptext