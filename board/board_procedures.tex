  \setlocalhsize \hsize\localhsize
  \startsimplecolumns
    {\bf Materials.}
\startitemize[1,joinedup]

  \startitem Computer (x86-64 architecture running a Linux distribution)

    \indenting[no] The software used runs on all platforms; a relatively
    recent (past 5 years or so) computer will suffice.
  \stopitem

  \startitem Computer Software: Python 3.3.3 with {\tt mpmath}, {\tt
    matplotlib}; XPP AUTO 7.0; \CONTEXT \ 2014.01.03; Gnuplot 4.6.4
  \stopitem
\stopitemize

{\bf Procedure.} XPP AUTO is used to compute a bifurcation diagram:

\midaligned{\showplot{bifurcation_xpp_simple}}

This shows, at a particular beta cell level, possible model states (T cell
levels). The higher the level, the more beta cells can be destroyed. On a
steady state, the system remains at a fixed T cell level, while the level
oscillates on a periodic state. Introducing a small deviation on a steady
state will result in the system returning to the original state, while in an
unstable state the system may move towards another state. \quotation{HB}
indicates a {\em Hopf bifurcation}, a point at which the system changes from
stable to periodic or vice-versa. For the {\em continuous} analysis, the
model starts at a particular $a_{15}$ level, which varies linearly over
time. At each time the $a_{15}$ level is unique, and thus the $A$ vs $t$
plot can be overlaid on the bifurcation diagram as a $A$ vs $a_{15}$ plot to
compare them (see {\em Results}).

The model equations are

{\switchtobodyfont[20pt]
\setupformulas[spacebefore=,spaceafter=,distance=0pt,margin=no]
\startformula\startalign
  \NC \frac{dA}{dt} \NC = (a₆+a₇M)f₁(p) - a₈A - a₉A² \;\;\;\;\; \frac{dM}{dt} = a_{10}\,f₂(p)A-f₁(p)a₇a_{16}M - a_{11}M \NR
  \NC \frac{dE}{dt} \NC = a_{12}(1-f₂(p))A-a_{13}E \;\;\;\;\; p =
  \frac{a_{14}}{a_{15}}EB \;\;\;\;\; f₁ = \frac{p^{a_1}}{{a_2}^{a_1}+p^{a_1}}  \;\;\;\;\; f₂ = \frac{{a_4}{a_5}^{a_3}}{{a_5}^{a_3}+p^{a_3}}\NR
\stopalign
\stopformula}

with $ a_{15} = δₚ = \mfunction{constant} $ for the bifurcation diagram or $
\frac{da₁₅}{dt} = \frac{dδₚ}{dt} = \mfunction{rate~of~change} $ for the
continuous analysis. Here, $A$ is the level of activated T cells, $B$
pancreatic beta cells, $E$ effector T cells, $M$ memory T cells, and $p$
peptide level. Beta cells that undergo apoptosis generate peptide; then, T
cells become activated, \quotation{recognizing} this peptide. These then
become either effector cells, which destroy pancreatic cells, or memory
cells, which trigger an immune response.

A 4th order Runge-Kutta solver (RK4) written in Python using {\tt mpmath} is
used to solve the continuous system. RK4 is standard for solving
differential equations \cite{Gonsalves2009}, used here because the
continuous analysis could lead to numerical precision errors
\cite{Baer1989}, and this implementation allows for arbitrary-precision
computations \cite{Mpmath2013}.
  \stopsimplecolumns