XPP AUTO, used in the original paper, is used to compute a bifurcation
diagram:

\showplot{bifurcation_xpp_simple}

This shows, at a particular level of beta cells, possible states of the
model (i.e.\ level of T cells). The higher the T cell level, the more beta
cells can be destroyed. On a steady state, the system remains at a fixed
level of T cells; in contrast, the level oscillates on a periodic state. If
a state is steady, then introducing a small deviation will result in the
system returning to the original state, while an unstable state will move
towards another state. The points labeled \quotation{HB} are {\em Hopf
  bifurcations}, points at which the system changes qualitative
behavior—from stable to periodic or vice-versa.

Now, for the {\em continuous} analysis, the model starts at a particular
$a_{15}$ (= 1/beta cell) level, which varies linearly over time. Because
this variation is linear, at each time the $a_{15}$ level is unique, and
thus the $A$ vs $t$ plot of this model can be overlaid on the bifurcation
diagram as a $A$ vs $a_{15}$ plot, allowing them to be compared (see the
diagrams below).

The model equations are

\setupformulas[spacebefore=,spaceafter=]
\startformula\startalign
  \frac{dA}{dt} \NC = \NC(a₆+a₇M)f₁(p) - a₈A - a₉A² \NR
  \frac{dM}{dt} \NC = \NC a_{10}f₂(p)A-f₁(p)a₇a_{16}M - a_{11}M \NR
  \frac{dE}{dt} \NC = \NC a_{12}(1-f₂(p))A-a_{13}E \NR
\stopalign\stopformula
\startformula
  p = \frac{a_{14}}{a_{15}}EB \;\;\;\;\; f₁ =
  \frac{p^{a_1}}{{a_2}^{a_1}+p^{a_1}} \;\;\;\;\; f₂ =
  \frac{{a_4}{a_5}^{a_3}}{{a_5}^{a_3}+p^{a_3}}
\stopformula

with $ a_{15} = δₚ = \mfunction{constant} $ for the bifurcation diagram or $
\frac{da₁₅}{dt} = \frac{dδₚ}{dt} = \mfunction{rate~of~change} $ for the
continuous analysis. Here, $A$ is the level of activated T cells, $B$
pancreatic beta cells, $E$ effector T cells, $M$ memory T cells, and $p$
peptide level. In the model, beta cells that undergo apoptosis (cell death)
generate peptide; then, T cells become activated, \quotation{recognizing}
this peptide. These then become either effector cells, which destroy
pancreatic cells and produce more peptide, or memory cells that encounter
the peptide and trigger an immune response.

A 4th order Runge-Kutta solver is used to solve the continuous
system. Runge-Kutta is a standard algorithm for solving differential
equations \cite{Gonsalves2009}, used here because the continuous analysis
could lead to numerical precision errors \cite{Baer1989}, and this
implementation allows for arbitrary-precision computations
\cite{Mpmath2013}.
