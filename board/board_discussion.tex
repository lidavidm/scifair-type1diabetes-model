To read the plots, note that each contains two graphs: the continuous model,
in orange, and the bifurcation diagram, in blue/green/red. The latter shows,
at a particular \math{\delta_p}, the level of T cells the static model
\quotation{settles into} over time. If multiple values are shown for a
particular \math{\delta_p}, then the system oscillates between those values
over time (and in this model, oscillations imply symptoms of diabetes
\cite{Mahaffy2007}). Meanwhile, the continuous model (orange) shows T cell
level vs.\ time; since \math{\delta_p} varies linearly with time, each
\math{\delta_p} corresponds to exactly one unique time value. The arrow in
the upper right indicates the direction of time.

In the original experiments with mice, the appearance of symptoms of
diabetes corresponded with cyclic fluctuations in the level of T cells
\cite{Trudeau2003}. Looking at the first plot, as the level of T cells
decreases over time in the continuous model, some oscillations appear before
the point they are expected to (the {\em Hopf bifurcation} point, labeled
\quotation{HB}). Given the large error bars in the original experiments,
however, such small oscillations may not be of note and may explain why only
a few oscillations were found in the mice. When the same amount of beta cell
destruction is spread out over a longer timeframe—1000 days rather than 200
(the experiments covered roughly 200 days)—the oscillations occur far later
than they should. Thus, theoretically, a way to prevent beta cell
destruction would only delay the onset of symptoms.
