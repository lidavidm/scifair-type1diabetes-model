To read the plots, note that each contains two graphs: the continuous model,
in orange, and the bifurcation diagram, in blue/green/red. The latter shows,
at a particular \math{\delta_p}, the level of T cells the static model
\quotation{settles into} over time. If multiple values are shown for a
particular \math{\delta_p}, then the system oscillates between those values
over time (and in this model, oscillations imply symptoms of diabetes
\cite{Mahaffy2007}). Meanwhile, the continuous model (orange), which is the
focus of this experiment, shows T cell level vs.\ time; since
\math{\delta_p} varies linearly with time, each \math{\delta_p} corresponds
to exactly one unique time value. The arrow in the upper right indicates the
direction of time.

In the original experiments with mice, the appearance of symptoms of
diabetes corresponded with cyclic fluctuations in the level of T cells
\cite{Trudeau2003}. (See the diagram in the introduction.)  Looking at
figure 1, as the level of T cells decreases over time in the continuous
model, some oscillations appear before the point they are expected to (the
{\em Hopf bifurcation} point, labeled \quotation{HB}). Given the large error
bars in the original experiments, however, such small oscillations may not
be of note and may explain why only a few oscillations were found
experimentally in the mice. In figure 2, when the same amount of beta cell
destruction occurs over a longer timeframe—1000 days rather than 200 (note
the experiments covered roughly 200 days)—the oscillations occur far later
than they should. This implies that theoretically, a way to slow down beta
cell destruction would delay the onset of symptoms. Meanwhile, in figure 3,
the beta cell level increases at a rate that outpaces the destruction caused
by the immune system. In this scenario, the immune system still displays the
response characteristic of the onset of diabetes—but as beta cell level is
increasing, this response does not matter. Figure 4 examines the other Hopf
bifurcation present, passing through it slowly; no oscillations occur, as
expected: a high \math{\delta_p} is normal \cite{Mahaffy2007} and increasing
it (or decreasing it; not shown) does not lead to disease.

\midaligned{\startcombination[2*1]
  {\externalfigure[precision_double][scale=600]}{}
  {\externalfigure[precision_100][scale=600]}{}
\stopcombination}

As for numerical precision, this experiment compared results from XPP and
Python; no qualitative difference was found. Further tests found that
running this model at double precision with certain parameter values leads
to incorrect results: at left is double precision, at right a higher
precision.
