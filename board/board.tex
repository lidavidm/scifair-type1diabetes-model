\definepapersize[board][width=48in,height=36in]
\setuppapersize[board][board]
\setuppagenumbering[state=stop]
\definebodyfontenvironment
  [default][b=1.3,c=2,d=2.1]
\definebodyfontenvironment[24pt]
\definebodyfontenvironment[28.8pt]
\definebodyfontenvironment[21.6pt]
\definebodyfontenvironment[19.2pt]
\definebodyfontenvironment[18pt]
\definebodyfontenvironment[16pt]
\definebodyfontenvironment[26pt]
\definebodyfontenvironment[31.2pt]
\definebodyfontenvironment[20.8pt]
\definebodyfontenvironment[19.2pt]
\definebodyfontenvironment[18pt]
\definebodyfontenvironment[16pt]

\definefontfamily[mainface][serif][TeX Gyre Pagella]
\definefontfamily[mainface][math][TeX Gyre Pagella Math]
\definefontfamily[mainface][mono][Consolas][scale=0.9]
\setupbodyfont[mainface,25pt]
\definefontfeature[smallcapitals][smcp=yes,kern=yes]
\definefontfamily[titleface][serif][Crimson][features=smallcapitals]
\setupalign[tolerant]

\input bibliography

\setupinterlinespace[1.3em]

\definelayer[title][x=12.5in,y=0in, width=23.5in, height=2in]
\definelayer[introduction][x=0.5in,y=3in,width=12in, height=32in]
\definelayer[procedures][x=12.5in,y=3in, width=15in, height=32in]
%\definelayer[results][x=24.5in,y=3in, width=15in, height=32in]
\definelayer[results][x=12.5in,y=14.7in, width=15in, height=32in]
\definelayer[resultsb][x=24.5in,y=14.7in, width=15in, height=32in]
\definelayer[discussion][x=36in,y=3in, width=12in, height=32in]

% https://kuler.adobe.com/TYPOGRAPHIC-CONTRAST-color-theme-3201506/
\definecolor[background][r=0.17,g=0.25,b=0.36]
\definecolor[foreground][r=1,g=1,b=1]
\definecolor[highlight][r=1,g=0.95,b=0.75]
\definecolor[bghighlight][r=0.9529412,g=0.5411765,b=0.4117647]

\input gnuplot_board
\input plots
\def\showplot#1{
  \startcolor[black]
    \switchtobodyfont[24pt]
    \framed[background=color,backgroundcolor=white]{
      \useGNUPLOTgraphic[#1]
    }
    \switchtobodyfont[26pt]
  \stopcolor
}

\def\sectiontitle#1{\startalignment[middle]{\switchtobodyfont[titleface]\scc #1}\stopalignment
  \blank[0.1in] \setupindenting[yes,next,0.5in]}


\setlayer[title][hoffset=0in,voffset=0in]{
  \framed[width=\dimexpr23in-6pt,height=3in,frame=off,foregroundcolor=foreground,background=color,backgroundcolor=bghighlight,bottomframe=on,leftframe=on,rightframe=on,framecolor=highlight,rulethickness=3pt]{
      {\bfd Enhancing Theoretical Understanding of the Onset of Type 1 Diabetes}
    \blank[0.4cm]
      {\tfb David Li}
    \blank[0.4cm]
      {\tfb Basha High School, 5990 South Val Vista Dr., Chandler, AZ 85249}
  }
}

  \setupTABLE[frame=off]
  \setupTABLE[row][offset=0in,toffset=0.3in]
  \setupTABLE[row][first][offset=0in,toffset=0in]
  \setupTABLE[c][distance=0em,width=15in]

\setlayer[introduction][hoffset=0in,voffset=0.25in]{
  \bTABLE
  \bTR
  \bTD
  \framed[width=\dimexpr11.5in-6pt,frame=on,align={normal,high},offset=0.3in,rulethickness=3pt,framecolor=highlight,foregroundcolor=foreground,background=color,backgroundcolor=background]{
    \sectiontitle{Introduction}

    \input board_introduction.tex

  }
  \eTD
  \eTR
  \bTR
  \bTD

  \framed[width=\dimexpr11.5in-6pt,frame=on,align={normal,high},offset=0.3in,rulethickness=3pt,framecolor=highlight,foregroundcolor=foreground,background=color,backgroundcolor=background]{
    \sectiontitle{Research Question}

    How does treating the peptide clearance rate $\delta_p$ (essentially,
    the level of pancreatic beta cells) as a {\em continuously} and slowly
    varying parameter affect the qualitative behavior of the scaled reduced
    immune model developed by Mahaffy and Edelstein-Keshet
    \cite{Mahaffy2007}, and how can those findings be applied to
    understanding and predicting the onset of type 1 diabetes?

  }
  \eTD
  \eTR
  \bTR
  \bTD

  \framed[width=\dimexpr11.5in-6pt,frame=on,align={normal,high},offset=0.3in,rulethickness=3pt,framecolor=highlight,foregroundcolor=foreground,background=color,backgroundcolor=background]{
    \sectiontitle{Hypothesis}

    If the model for the level of immune cells in the weeks before the onset
    of type 1 diabetes is analyzed with both a continuously varying and a
    static peptide clearance rate $δₚ$, then in the former analysis, the
    oscillations present in the original model that indicate the onset of
    diabetes will begin later than in the latter model because previous
    research has shown this behavior is delayed in other systems when
    similarly analyzed with a continuously varying parameter.

  }
  \eTD
  \eTR
  \eTABLE
}

\setlayer[procedures][hoffset=0in,voffset=0.25in]{

  \framed[width=\dimexpr23in-6pt,frame=on,align={normal,high},offset=0.3in,rulethickness=3pt,framecolor=highlight,foregroundcolor=foreground,background=color,backgroundcolor=background]{
    \sectiontitle{Experimental (Materials \& Procedures)}

    \input board_procedures
  }
%   \bTABLE
% \bTR
%   \bTD

%   \framed[width=\dimexpr11in-6pt,frame=on,align={normal,high},offset=0.3in,rulethickness=3pt,framecolor=highlight,foregroundcolor=foreground,background=color,backgroundcolor=background]{
%     \sectiontitle{Experimental (Materials)}

%     \input board_materials
%   }
%   \eTD
%   \eTR
%   \bTR
%   \bTD
%   \framed[width=\dimexpr11in-6pt,frame=on,align={normal,high},offset=0.3in,rulethickness=3pt,framecolor=highlight,foregroundcolor=foreground,background=color,backgroundcolor=background]{
%     \sectiontitle{Experimental (Procedures)}
%     \input board_procedures
%   }
%   \eTD
%   \eTR

%   \bTR
%   \bTD

%   \eTD
%   \eTR
%   \eTABLE
}

\setlayer[results][hoffset=0in,voffset=0.25in]{
  % \bTABLE
  % \bTR
  % \bTD
  \framed[width=\dimexpr11in-6pt,frame=on,align={normal,high},offset=0.3in,rulethickness=3pt,framecolor=highlight,foregroundcolor=foreground,background=color,backgroundcolor=background]{
    \sectiontitle{Results}
    \showplot{bifurcation_combined_12}
    \startalignment[middle]
      {\bf Figure 1.} Combined diagram showing T cell level over time (as beta cell level decreases), over a duration of 200 days
    \stopalignment

    \showplot{bifurcation_combined_12_long}
    \startalignment[middle]
      {\bf Figure 2.} Combined diagram showing T cell level over time (as beta cell level decreases), but over a duration of 1000 days
    \stopalignment

    \showplot{bifurcation_combined_21}
    \startalignment[middle]
      {\bf Figure 3.} Combined diagram showing T cell level over time (as beta cell level {\em increases}), over a duration of 200 days
    \stopalignment

    \showplot{bifurcation_combined_h2_forward}
    \startalignment[middle]
      {\bf Figure 4.} Combined diagram showing T cell level over time (as
      beta cell level {\em increases}), over a duration of 3200 days
    \stopalignment

    \indenting[no]

      Not all parameter ranges tested are shown for lack of space


  }
  % \eTD
  % \eTR

  % \bTR
  % \bTD
}

\setlayer[resultsb][hoffset=0in,voffset=0.25in]{
  \framed[width=\dimexpr11in-6pt,frame=on,align={normal,high},offset=0.3in,rulethickness=3pt,framecolor=highlight,foregroundcolor=foreground,background=color,backgroundcolor=background]{
    \sectiontitle{Discussion of Results}

    \input board_discussion
  }
  % \eTR
  % \eTD
  % \eTABLE
}

\setlayer[discussion][hoffset=0in,voffset=0.25in]{
  \bTABLE

  \bTR
  \bTD
  \framed[width=\dimexpr11.5in-6pt,frame=on,align={normal,high},offset=0.3in,rulethickness=3pt,framecolor=highlight,foregroundcolor=foreground,background=color,backgroundcolor=background]{
    \sectiontitle{Conclusion}

    \input board_conclusion
  }
  \eTR
  \eTD

  \bTR
  \bTD
  \framed[width=\dimexpr11.5in-6pt,frame=on,align={normal,high},offset=0.3in,rulethickness=3pt,framecolor=highlight,foregroundcolor=foreground,background=color,backgroundcolor=background]{
    \sectiontitle{Future Research}

    \input board_future_research
  }
  \eTD
  \eTR

  \bTR
  \bTD
  \framed[width=\dimexpr11.5in-6pt,frame=on,align={normal,high},offset=0.3in,rulethickness=3pt,framecolor=highlight,foregroundcolor=foreground,background=color,backgroundcolor=background]{
    \sectiontitle{Works Cited}
    \switchtobodyfont[22pt]
    \setupalign[nothyphenated,verytolerant]
    \placebibliography{summary.bib}
    \setupalign[hyphenated]
  }
  \eTR
  \eTD

  \bTR
  \bTD
  \framed[width=\dimexpr11.5in-6pt,frame=on,align={normal,high},offset=0.3in,rulethickness=3pt,framecolor=highlight,foregroundcolor=foreground,background=color,backgroundcolor=background]{
    \sectiontitle{Acknowledgements}

      Thanks to Dr.\ Michael McKelvy, Dr.\ Stephen Baer, Dr.\ Joseph
      Mahaffy, Dr.\ Jiaxu Li

  }
  \eTR
  \eTD
  \eTABLE
}

\setupbackgrounds[page][background={color,title,introduction,procedures,results,resultsb,discussion},backgroundcolor=background]
\starttext

~

\stoptext