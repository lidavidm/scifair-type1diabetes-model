\setupindenting[yes,next,0.5in]

In type 1 diabetes (autoimmune diabetes), the immune system attacks and
destroys insulin-producing pancreatic beta cells \cite{PubMed2013}.
Currently, the causes and cures are largely unknown \cite{Daneman2006}.
However, a previous experiment has shown that in diabetic mice, the level of
T cells fluctuates cyclically in the weeks before the appearance of symptoms
(graph, below left): after the \quotation{spikes} occur (black), the
percentage of diabetic mice increases dramatically (in gray)
[\cite{Trudeau2003} cited in \cite{Mahaffy2007}]. To better understand the
underlying mechanism, Mahaffy and Edelstein-Keshet constructed a
mathematical model of the immune–pancreas system. One parameter in the model
is the pancreatic beta cell level, which slowly decreases over time; at a
certain level, the fluctuations described experimentally occur, and thus
then diabetes appears \cite{Mahaffy2007}.

\setupcaptions[location=bottom,width=10in]
\placefigure[here,nonumber]{\switchtobodyfont[20pt]Left: from \cite{Mahaffy2007}; experimental
  measurement of T cell levels (black) in NOD mice over time, with diabetes
  occurring after \quotation{spikes} (percentage of mice that are diabetic
  in gray). Right: example of a Hopf bifurcation.}{\startcombination[2*1]
  {\externalfigure[experiment][height=8cm]}{}
  {\externalfigure[bifurcation][height=8cm]}{}
\stopcombination}


Originally, the researchers analyzed the model’s behavior at various
constant parameter values \cite{Mahaffy2007}; they searched for {\em
  bifurcations}, or qualitative changes in behavior that occur when a
parameter reaches a certain threshold \cite{VanVoorn2006}. For instance, a
system may remain constant at one parameter value; at another level, the
system may oscillate between two defined values (see diagram, above right)
\cite{VanVoorn2006}. This experiment improves Mahaffy and Edelstein-Keshet’s
model by applying research demonstrating that in certain systems, slowly
varying the parameter value while the model runs can change the system
behavior qualitatively \cite{Baer1989}. This more accurately reflects what
occurs biologically, as the original paper states the parameter continuously
slowly falls. To summarize: for the original {\em static} analysis, the
authors re-ran the model multiple times, each time setting the parameter to
a fixed value. For the {\em continuous} analysis here, the parameter starts
at a value and then continuously changes over time. Thus, this experiment
will help scientists and medical professionals better understand and apply
the theoretical results of the model as well as the experimental results
cited in predicting and understanding the onset of type 1 diabetes and the
behavior of the immune system in this disease.

Furthermore, this experiment will investigate the use of {\tt mpmath} in
investigating such models. Previous research demonstrated that numerical
error due to rounding can lead to incorrect results \cite{Baer1989}; {\tt
  mpmath} allows the user to set any desired precision to avoid such errors.
