\setupindenting[yes,next,0.5in]

In type 1 diabetes, also known as autoimmune diabetes, the body’s own immune
system attacks and destroys insulin-producing pancreatic beta cells, leading
to an insulin shortage and causing symptoms \cite{PubMed2013}. Currently,
the causes and cures are largely unknown \cite{Daneman2006}. However,
previous experiments have shown that in NOD (non-obese diabetic) mice, a
standard model for diabetic research, the level of T cells (a specific type
of immune cell) fluctuates cyclically in the weeks leading up to the
appearance of symptoms, as depicted in the graph (below left): after the
\quotation{spikes} occur (black), the percentage of diabetic mice increases
dramatically (in gray) [\cite{Trudeau2003} cited in \cite{Mahaffy2007}]. To
better understand the mechanism underlying these oscillations, Mahaffy and
Edelstein-Keshet constructed a mathematical model of the immune–pancreas
system. One parameter in the model is the level of pancreatic beta cells,
which slowly decreases over time; at a certain level, the fluctuations
described experimentally occur, and then diabetes appears
\cite{Mahaffy2007}.


\setupcaptions[location=bottom,width=10in]
\placefigure[here,nonumber]{\switchtobodyfont[20pt]Left: from \cite{Mahaffy2007}; experimental
  measurement of T cell levels (black) in NOD mice over time, with diabetes
  occurring after \quotation{spikes} (percentage of mice that are diabetic
  in gray). Right: example of a Hopf bifurcation.}{\startcombination[2*1]
  {\externalfigure[experiment][height=8cm]}{}
  {\externalfigure[bifurcation][height=8cm]}{}
\stopcombination}


Originally, the researchers analyzed the model’s behavior at various
constant parameter values \cite{Mahaffy2007}; in particular, they searched
for {\em bifurcations}, or qualitative changes in behavior that occur when a
parameter reaches a certain threshold \cite{VanVoorn2006}. For instance, a
system may remain constant at one parameter value; if the parameter reaches
a certain level, the system may then oscillate between two defined values
(see diagram) \cite{VanVoorn2006}. This experiment will improve the model by
applying research demonstrating that in certain systems, slowly varying the
parameter value while the model runs can change the qualitative nature of
the system \cite{Baer1989}. This more accurately reflects what occurs
biologically, as the original paper explicitly stated that the parameter
should continuously slowly fall. To summarize: for the original {\em static}
analysis, the authors re-ran the model multiple times, each time setting the
parameter to a fixed value. For the {\em continuous} analysis here, the
parameter starts at a given value and then continuously decreases over
time. Thus, this experiment will help scientists and medical professionals
better understand and apply the theoretical results of Mahaffy and
Edelstein-Keshet’s work as well as the experimental results they cited in
predicting and understanding the onset of type 1 diabetes and the behavior
of the immune system in this disease.