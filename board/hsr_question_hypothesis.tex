\environment env_simple

\starttext

\title{Question \& Hypothesis}
David Li

Question: How does treating the peptide clearance rate $\delta_p$
(essentially, the level of pancreatic beta cells) as a {\em continuously}
and slowly varying parameter affect the qualitative behavior of the scaled
reduced immune model developed by Mahaffy and Edelstein-Keshet, and how can
these findings be applied to understanding and predicting the onset of type
1 diabetes?

Hypothesis: If the model for the level of immune cells in the weeks before
the onset of type 1 diabetes is analyzed with both a continuously varying
and a static peptide clearance rate $δₚ$, then in the former analysis, the
oscillations present in the original model that indicate the onset of
diabetes will begin later than in the latter model because previous research
has shown such behavior is delayed in other systems when similarly analyzed
with a continuously varying parameter.

\stoptext