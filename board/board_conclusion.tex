The results both confirm and refute the hypothesis—depending on the
timescale, the oscillations that are indicative of the onset of diabetes may
start either earlier or later than expected. Taking into account the error
bars on the original data, the simulations here still show the
characteristic pattern of low T cell levels that then transition into
oscillatory \quotation{spiking}. Thus, this confirms the veracity of the
original model under a more biologically accurate continuous
analysis. However, the results herein further suggest that 1) with sensitive
enough tests, the oscillations characteristic of the disease may be
detectable earlier than expected, and that 2) slowing the rate of beta cell
destruction may delay, but not prevent, those oscillations. Therefore,
researchers interested in completely preventing this disease may want to
investigate other variables of the model, and being able to theoretically
detect the onset of type 1 diabetes earlier than expected before would allow
more time for treatment and hopefully a higher quality of life for those
affected.

Experimental error in this model may result from round-off due to finite
numerical precision. To address this, {\tt mpmath} with Python allowed the
calculations to be conducted at higher precision for comparison with the
result from XPP. Multiple trials were not necessary as the calculations were
deterministic. (Internally, {\tt mpmath} represents values with an integer
mantissa and exponent and guarantees the correctness of basic
arithmetic operations, the only ones used here \cite{Mpmath2013}.) This
experiment thus demonstrated the viability of Python in research by
comparing it to the standard tool XPP and showing when it is more capable.
