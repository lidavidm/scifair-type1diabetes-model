The results both confirm and refute the hypothesis—depending on the
timescale, the oscillations that are indicative of the onset of diabetes may
start either earlier or later than expected. Taking into account the error
bars on the original data, the simulations here still show the
characteristic pattern of low T cell levels that then transition into
oscillatory \quotation{spiking}. Thus, this confirms the veracity of the
original model under a more biologically accurate continuous
analysis. Furthermore, the results suggest that 1) with sensitive enough
tests, the oscillations characteristic of the disease may be detectable
earlier than expected, and that 2) slowing the rate of beta cell destruction
may delay, but not prevent, those oscillations. Therefore, researchers
interested in completely preventing this disease may want to investigate
other variables of the model.
