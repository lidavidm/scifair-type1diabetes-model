\environment env_simple
\usemodule[gnuplot]

\setupGNUPLOT[terminal=context,options={size 6in,2in}]
\startGNUPLOTinclusions
  set datafile separator " ";
  set key inside right top;
  set style line 1 lt 1 lc rgb "#1f77b4" lw 2 pt 0 ps 1;
  set style line 13 lt 3 lc rgb "#1f77b4" lw 2 pt 0 ps 1;
  set style line 14 lt 1 lc rgb "#1f77b4" lw 2 pt 0 ps 1;
  set style line 15 lt 1 lc rgb "#1f77b4" lw 2 pt 6 ps 1;
  set style line 2 lt 1 lc rgb "#ff7f0e" lw 2;
  set style line 3 lt 1 lc rgb "#2ca02c" lw 2 pt 0 ps 1;
  set style line 4 lt 1 lc rgb "#d62728" lw 2 pt 0 ps 1;
  set style line 43 lt 1 lc rgb "#d62728" lw 2 pt 6 ps 1;
  set style arrow 12 ls 2;
  set style arrow 16 ls 2 lw 3;
  set style line 80 lt rgb "#000000";
  set style line 81 lt 0 lw 3;
  set style line 81 lt rgb "#999999";
  set xlabel offset 0,-1;
  set ylabel offset -1,0;
\stopGNUPLOTinclusions

\startGNUPLOTscript[bifurcation_xpp_simple]
  set xlabel "$a_{15} = \\delta_p = $\\chardef\\vulgarfractionmethod2\\vfrac1{beta~cell~level}";
  set ylabel "$A = $ T Cells ($\\times 10^3$)";
  set arrow from 0.5707,0.7 to 0.5707,0.3 as 12;
  set label "HB" at 0.5707,1 center;
  set arrow from 4.063,1.2 to 4.063,0.8 as 12;
  set label "HB" at 4.063,1.5 center;
  set xrange [0:30];
  set key inside right center;
  plot "mk_static.dat.1" using 1:($3==1?$2:1/0) title "Stable steady state" with lines ls 14,\
       "mk_static.dat.1" using 1:($3==1?1/0:$2) title "Unstable steady state" with lines ls 13,\
       "mk_static.dat.2" using 1:($3==3?$2:1/0) title "Stable periodic" with points ls 3,\
       "mk_static.dat.2" using 1:($3==3?1/0:$2) title "Unstable periodic" with points ls 43,\
       "mk_static.dat.3" using 1:($3==3?$2:1/0) notitle with points ls 3,\
       "mk_static.dat.3" using 1:($3==3?1/0:$2) notitle with points ls 43,\
       "mk_static.dat.4" using 1:($3==3?$2:1/0) notitle with points ls 3,\
       "mk_static.dat.4" using 1:($3==3?1/0:$2) notitle with points ls 43,\
       "mk_static.dat.5" using 1:($3==3?$2:1/0) notitle with points ls 3,\
       "mk_static.dat.5" using 1:($3==3?1/0:$2) notitle with points ls 43;
\stopGNUPLOTscript
\def\showplot#1{
    \framed[background=color,backgroundcolor=white]{
      \useGNUPLOTgraphic[#1]
    }
}
\starttext
\title{Introduction}

\indenting[no] David Li \indenting[yes]

\input bibliography
\startitemize[1,joinedup]

  \startitem Computer (x86-64 architecture running a Linux distribution)

    \indenting[no] The exact specifications and the operating system do not
    matter, as the software used runs on all platforms. A relatively recent
    (past 5 years or so) computer will suffice.
  \stopitem

  \startitem Computer Software
    \startitemize[1,joinedup]
    \item Python 3.3.3 with {\tt mpmath}
      (\hyphenatedurl{http://www.python.org},
      \hyphenatedurl{http://www.mpmath.org}), used to calculate the data for
      the plots.
    \item XPP AUTO 7.0
      (\hyphenatedurl{http://www.math.pitt.edu/~bard/xpp/xpp.html}), used to
      calculate the data for the bifurcation diagrams.
    \item \CONTEXT \ 2014.01.03 (http://wiki.contextgarden.net), used to
      generate the plots and bifurcation diagrams.
    \item Gnuplot 4.6.4 (http://www.gnuplot.info), used to generate
      plots.
    \stopitemize
  \stopitem
\stopitemize

XPP AUTO, used in the original paper, is used to compute a bifurcation
diagram:

\showplot{bifurcation_xpp_simple}

This shows, at a particular level of beta cells, possible states of the
model (i.e.\ level of T cells). The higher the T cell level, the more beta
cells can be destroyed. On a steady state, the system remains at a fixed
level of T cells; in contrast, the level oscillates on a periodic state. If
a state is steady, then introducing a small deviation will result in the
system returning to the original state, while an unstable state will move
towards another state. The points labeled \quotation{HB} are {\em Hopf
  bifurcations}, points at which the system changes qualitative
behavior—from stable to periodic or vice-versa.

Now, for the {\em continuous} analysis, the model starts at a particular
$a_{15}$ (= 1/beta cell) level, which varies linearly over time. Because
this variation is linear, at each time the $a_{15}$ level is unique, and
thus the $A$ vs $t$ plot of this model can be overlaid on the bifurcation
diagram as a $A$ vs $a_{15}$ plot, allowing them to be compared (see the
diagrams below).

The model equations are

\setupformulas[spacebefore=,spaceafter=]
\startformula\startalign
  \frac{dA}{dt} \NC = \NC(a₆+a₇M)f₁(p) - a₈A - a₉A² \NR
  \frac{dM}{dt} \NC = \NC a_{10}f₂(p)A-f₁(p)a₇a_{16}M - a_{11}M \NR
  \frac{dE}{dt} \NC = \NC a_{12}(1-f₂(p))A-a_{13}E \NR
\stopalign\stopformula
\startformula
  p = \frac{a_{14}}{a_{15}}EB \;\;\;\;\; f₁ =
  \frac{p^{a_1}}{{a_2}^{a_1}+p^{a_1}} \;\;\;\;\; f₂ =
  \frac{{a_4}{a_5}^{a_3}}{{a_5}^{a_3}+p^{a_3}}
\stopformula

with $ a_{15} = δₚ = \mfunction{constant} $ for the bifurcation diagram or $
\frac{da₁₅}{dt} = \frac{dδₚ}{dt} = \mfunction{rate~of~change} $ for the
continuous analysis. Here, $A$ is the level of activated T cells, $B$
pancreatic beta cells, $E$ effector T cells, $M$ memory T cells, and $p$
peptide level. In the model, beta cells that undergo apoptosis (cell death)
generate peptide; then, T cells become activated, \quotation{recognizing}
this peptide. These then become either effector cells, which destroy
pancreatic cells and produce more peptide, or memory cells that encounter
the peptide and trigger an immune response.

A 4th order Runge-Kutta solver is used to solve the continuous
system. Runge-Kutta is a standard algorithm for solving differential
equations \cite{Gonsalves2009}, used here because the continuous analysis
could lead to numerical precision errors \cite{Baer1989}, and this
implementation allows for arbitrary-precision computations
\cite{Mpmath2013}.

\placebibliography{summary.bib}
\stoptext
