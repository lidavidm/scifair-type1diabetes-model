\environment env_hsr

\starttext

\title{Application of the Slow Passage through a Hopf Bifurcation to Models
  for Arctic Sea Ice}

David Li

\blank[0.5cm]

\noindentation

The melting of Arctic sea ice has concerned climate change scientists due to
its potential to greatly impact weather patterns in the United States and
Europe. For instance, Greene and Monger cite evidence that links severe
winter weather in the U.S.\ and Europe to the decline in summer sea ice
\cite{Greene2012}. Such weather is controlled by the {\em Arctic
  Oscillation} (AO), which in turn depends on the levels of sea ice. With
lower ice levels in the summer, more of the ocean is exposed to solar
radiation; the heat stored in the ocean warms the atmosphere during autumn,
leading to conditions
that allow cold Arctic air to affect lower
latitudes—resulting in a cold winter. According to Greene and Monger,
forecasters generally didn’t account for the AO when predicting weather
because of its unpredictability, but the melting sea ice clearly caused a
general trend in its behavior which matters in the long run. Arctic sea ice,
rather than being a static system, changes continuously; in particular,
levels of ice vary with the seasons \cite{Polyak2010}. A lack of sea ice,
beyond the consequences described above, could lead to more coastal erosion
and affect Atlantic current circulation, also affecting European and
American weather. Clearly, the ability to accurately model the levels of ice
in the Arctic matters to those studying this climate system and its
consequences.

Sea ice modeling applies physics and assumptions to approximate the behavior
of the system \cite{Bitz2010}. Oftentimes such models are developed as
components of a more comprehensive global climate model (GCM) that also
simulates the atmosphere and ocean. Alternatively, simpler, mathematical
models that incorporate various physical parameters have been studied, such
as the one examined by Abbot et.\ al.\ \cite{Abbot2011}. Such models, which
often consist of a few differential equations, are particularly interesting
because of the possibility of bifurcations, or qualitative changes in the
behavior of the system. Finding a bifurcation mathematically may raise the
possibility of a real-world equivalent; in the case of sea ice, this may
mean that beyond a certain level of greenhouse gas in the atmosphere, the
Arctic ice system will suddenly transition to an ice-free state, rather than
gradually shrinking over the years. Abbot’s study found that some such
bifurcations existed in their model, but found that GCMs generally did not
exhibit the same behavior in most cases.

One type of bifurcation which Abbot et.\ al.\ did not study is known as a
Hopf bifurcation. However, they can exhibit interesting behavior—notably,
Baer et.\ al.\ studied their presence in a model for neurons and found that
in some cases, such bifurcations could lead to erroneous results in the
algorithms used to numerically approximate the solution to the model
\cite{Baer1989}. In particular, such behavior occurred if the system used a
parameter that slowly rose over time—in the case of a climate model, the
level of greenhouse gases would be an example. Therefore, it would be
important to know whether this also applies to the mathematical model
discussed by Abbot et.\ al., as that study incremented greenhouse gas levels
in discrete steps, but did not consider a continuously varying parameter as
Baer did; furthermore, studying bifurcations in simpler models can help
researchers recognize them in the more complicated GCMs
\cite{Abbot2011}. Therefore, this leads to the question: {\em how does the
  level of greenhouse gas as a slow parameter affect the behavior of an
  Arctic sea ice model, and how can those conclusions be applied to a
  general climate model?}

\blank[0.5cm]

\section{References}

\placebibliography{summary.bib}

\stoptext
%%% Local Variables:
%%% mode: context
%%% TeX-master: t
%%% End: